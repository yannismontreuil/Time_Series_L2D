%! Author = yannis
%! Date = 09/02/2026

\documentclass[11pt]{article}
\usepackage{amsmath,amssymb}
\usepackage{mathtools}

\title{Why the Tri-Cycle Correlation Setting Favors the Factorized Router}
\author{}
\date{}

\begin{document}
\maketitle

\section{Goal}
We want a synthetic setting where a router that models a shared latent factor
$g_t$ and regimes $z_t$ can strictly outperform (i) baselines that treat experts
as conditionally independent given the context and (ii) the same router with
$g_t$ removed. This note explains, from the model in \texttt{factorized\_router.tex},
how the configuration in \texttt{config/config\_tri\_cycle\_corr\_v3.yaml} is
designed to meet that goal.

\section{Model and information advantage}
Under the factorized SLDS, residuals follow
\[
e_{t,k} = \phi(x_t)^\top (B_k g_t + u_{t,k}) + \varepsilon_{t,k}, \qquad
\varepsilon_{t,k} \sim \mathcal{N}(0, R_{z_t,k}).
\]
When $g_t$ is informative and shared across experts, an observation of one
expert immediately reduces uncertainty about \emph{other} experts. The router
exploits this via the $g$-information term (and optionally the $z$-information
term), which is absent in baselines that do not model the shared factor.
In contrast, a router without $g_t$ is forced to treat each expert as having
independent dynamics, losing cross-expert transfer.

\section{Design principles for dominance}
The v3 configuration is chosen to amplify exactly the information that the
factorized model can use, while making it hard for baselines that rely only on
per-expert losses.

\paragraph{(P1) Strong shared signal, weak idiosyncratic noise.}
We increase the shared perturbation scale and decrease idiosyncratic noise:
\[
\sigma_{\text{shared}} \gg \sigma_{\text{indiv}}.
\]
Then, conditional on $z_t$, the residuals of the correlated pair are dominated
by the same latent $g_t$. This yields a high mutual information
$\mathcal{I}(g_t; e_{t,k}\mid \mathcal{F}_t, z_t)$ and makes the cross-update
in Proposition~\ref{prop:cross_update} effective. Baselines that treat experts
independently cannot exploit this coupling and incur higher expected cost.

\paragraph{(P2) Regime-dependent correlation structure.}
The correlated pairs rotate with $z_t$. This makes static correlations
misleading and forces correct regime tracking. The factorized model carries
both $z_t$ and $g_t$ and therefore can adapt the correlation structure online,
whereas a no-$g_t$ model has to relearn each expert independently after every
regime shift.

\paragraph{(P3) Short blocks and repeated switches.}
Shorter regime blocks increase non-stationarity. This penalizes methods that
need long stationary windows to fit per-expert parameters (e.g., L2D, UCB),
while the factorized model can transfer information quickly across experts
through $g_t$ and can exploit the IMM structure for fast regime adaptation.

\paragraph{(P4) Clear separation of ``best'' experts per regime.}
We set biases so that the correlated pair is the clear winner in its regime.
Thus, when the model identifies the current pair, it can immediately exploit
low cost; baselines that hesitate or over-explore pay larger regret.

\paragraph{(P5) Lower observation noise on $y_t$.}
Reducing the AR(1) observation noise increases the signal-to-noise ratio in
the residuals. This benefits the latent-factor model because $g_t$ becomes
more identifiable from a single observation.

\section{Why the no-$g_t$ router underperforms}
The no-$g_t$ model removes the shared factor, leaving only idiosyncratic states
$u_{t,k}$. In the v3 setting, the true residuals are dominated by shared noise,
so the no-$g_t$ model must attribute shared variation to independent $u_{t,k}$.
This inflates predictive variance and breaks the transfer across experts.
As a result, the no-$g_t$ router needs more samples per expert per regime,
which is unavailable under partial feedback with regime switches.

\section{Why baselines underperform}
Baselines such as L2D, L2D\_SW, LinUCB, and NeuralUCB do not model the shared
latent factor explicitly. They therefore cannot update the uncertainty of an
unobserved expert using the observed residual of a correlated expert. Under
the v3 setting:
\begin{itemize}
\item The shared factor is strong, so ignoring it loses a large information
      advantage.
\item Regimes switch repeatedly, so per-expert learning has to restart often.
\item The correlated pair is only identifiable through cross-expert transfer,
      which baselines lack.
\end{itemize}

\section{Expected outcome}
In this regime, the factorized router with $g_t$ should:
\begin{enumerate}
\item learn the shared structure quickly under partial feedback,
\item track the regime-dependent correlation map,
\item select the correct expert pair earlier and more consistently,
\item produce lower average cost than baselines and the no-$g_t$ variant.
\end{enumerate}

This is a \emph{design} argument, not a formal guarantee. However, it aligns
directly with the information-theoretic rationale of the model and the
mechanisms in \texttt{factorized\_router.tex}.

\end{document}
