%%%%%%%% ICML 2026 EXAMPLE LATEX SUBMISSION FILE %%%%%%%%%%%%%%%%%

\documentclass{article}

% Recommended, but optional, packages for figures and better typesetting:
\usepackage{microtype}
\usepackage{graphicx}
\usepackage{subcaption}
\usepackage{booktabs} % for professional tables
\usepackage{longtable} % for notation table in appendix

% hyperref makes hyperlinks in the resulting PDF.
% If your build breaks (sometimes temporarily if a hyperlink spans a page)
% please comment out the following usepackage line and replace
% \usepackage{icml2026} with \usepackage[nohyperref]{icml2026} above.
\usepackage{hyperref}


% Attempt to make hyperref and algorithmic work together better:
\newcommand{\theHalgorithm}{\arabic{algorithm}}

% Use the following line for the initial blind version submitted for review:
\usepackage{icml2026}

% For preprint, use
% \usepackage[preprint]{icml2026}

% If accepted, instead use the following line for the camera-ready submission:
% \usepackage[accepted]{icml2026}

\renewcommand{\bibname}{References}
\renewcommand{\bibsection}{\subsubsection*{\bibname}}

% If you use BibTeX in apalike style, activate the following line:
% \bibliographystyle{apalike}

% \usepackage[hidelinks]{hyperref}
\newcommand{\fix}{\marginpar{FIX}}
\newcommand{\new}{\marginpar{NEW}}
% \newtheorem{theorem}{Theorem}
\usepackage{algorithm}
\let\oldemptyset\emptyset
\let\emptyset\varnothing
\usepackage{algorithmic}
\usepackage{amsfonts}
\usepackage{amsthm}
\usepackage{thmtools, thm-restate}
\newtheorem{assumption}{Assumption}
\newcommand{\Pm}{\mathbb{P}}
\newcommand{\mc}{\mathcal}
\newcommand{\mb}{\mathbb}
\newcommand{\EM}{\text{EM}}
\newcommand{\vy}{\mathbf{y}}
\newcommand{\vhaty}{\widehat{\vy}}
\newcommand{\owt}[1]{\textcolor{blue}{#1}}
\usepackage{amsmath}
\DeclareMathOperator*{\argmax}{arg\,max}
\DeclareMathOperator*{\argmin}{arg\,min}
\DeclareMathOperator*{\argsup}{arg\,sup}
\DeclareMathOperator*{\arginf}{arg\,inf}
\let\emptyset\varnothing
\usepackage{appendix}
\usepackage{titletoc}
\newcommand\DoToC{%
  \startcontents
  \printcontents{}{2}{\textbf{Contents}\vskip3pt\hrule\vskip5pt}
  \vskip3pt\hrule\vskip5pt
}
\usepackage{amssymb}
\usepackage{tikz}
\usetikzlibrary{arrows.meta,positioning,calc,fit,shapes.misc,shapes.geometric, calc, bayesnet}
% if you use cleveref..
\usepackage[capitalize,noabbrev]{cleveref}
\usepackage{enumitem}
%%%%%%%%%%%%%%%%%%%%%%%%%%%%%%%%
% THEOREMS
%%%%%%%%%%%%%%%%%%%%%%%%%%%%%%%%
\newcommand{\BlackBox}{\rule{1.5ex}{1.5ex}}  % end of proof
\ifdefined\proof
    \renewenvironment{proof}{\par\noindent{\bf Proof\ }}{\hfill\BlackBox\\[2mm]}
\else
    \newenvironment{proof}{\par\noindent{\bf Proof\ }}{\hfill\BlackBox\\[2mm]}
\fi
\newtheorem{example}{Example} 
\newtheorem{theorem}{Theorem}
\newtheorem{lemma}[theorem]{Lemma} 
\newtheorem{proposition}[theorem]{Proposition} 
\newtheorem{remark}[theorem]{Remark}
\newtheorem{corollary}[theorem]{Corollary}
\newtheorem{definition}[theorem]{Definition}
\newtheorem{conjecture}[theorem]{Conjecture}
\newtheorem{axiom}[theorem]{Axiom}
\usepackage{mathtools}

% Todonotes is useful during development; simply uncomment the next line
%    and comment out the line below the next line to turn off comments
%\usepackage[disable,textsize=tiny]{todonotes}
\usepackage[textsize=tiny]{todonotes}

% The \icmltitle you define below is probably too long as a header.
% Therefore, a short form for the running title is supplied here:
\icmltitlerunning{Learning to Defer in Non-Stationary Time Series via Switching State-Space Models}

\begin{document}

\twocolumn[
  \icmltitle{Learning to Defer in Non-Stationary Time Series \\
  via Switching State-Space Models}

  % It is OKAY to include author information, even for blind submissions: the
  % style file will automatically remove it for you unless you've provided
  % the [accepted] option to the icml2026 package.

  % List of affiliations: The first argument should be a (short) identifier you
  % will use later to specify author affiliations Academic affiliations
  % should list Department, University, City, Region, Country Industry
  % affiliations should list Company, City, Region, Country

  % You can specify symbols, otherwise they are numbered in order. Ideally, you
  % should not use this facility. Affiliations will be numbered in order of
  % appearance and this is the preferred way.
  \icmlsetsymbol{equal}{*}

  \begin{icmlauthorlist}
    \icmlauthor{Firstname1 Lastname1}{equal,yyy}
    \icmlauthor{Firstname2 Lastname2}{equal,yyy,comp}
    \icmlauthor{Firstname3 Lastname3}{comp}
    \icmlauthor{Firstname4 Lastname4}{sch}
    \icmlauthor{Firstname5 Lastname5}{yyy}
    \icmlauthor{Firstname6 Lastname6}{sch,yyy,comp}
    \icmlauthor{Firstname7 Lastname7}{comp}
    %\icmlauthor{}{sch}
    \icmlauthor{Firstname8 Lastname8}{sch}
    \icmlauthor{Firstname8 Lastname8}{yyy,comp}
    %\icmlauthor{}{sch}
    %\icmlauthor{}{sch}
  \end{icmlauthorlist}

  \icmlaffiliation{yyy}{Department of XXX, University of YYY, Location, Country}
  \icmlaffiliation{comp}{Company Name, Location, Country}
  \icmlaffiliation{sch}{School of ZZZ, Institute of WWW, Location, Country}

  \icmlcorrespondingauthor{Firstname1 Lastname1}{first1.last1@xxx.edu}
  \icmlcorrespondingauthor{Firstname2 Lastname2}{first2.last2@www.uk}

  % You may provide any keywords that you find helpful for describing your
  % paper; these are used to populate the "keywords" metadata in the PDF but
  % will not be shown in the document
  \icmlkeywords{Machine Learning, ICML}

  \vskip 0.3in
]

% this must go after the closing bracket ] following \twocolumn[ ...

% This command actually creates the footnote in the first column listing the
% affiliations and the copyright notice. The command takes one argument, which
% is text to display at the start of the footnote. The \icmlEqualContribution
% command is standard text for equal contribution. Remove it (just {}) if you
% do not need this facility.

% Use ONE of the following lines. DO NOT remove the command.
% If you have no special notice, KEEP empty braces:
\printAffiliationsAndNotice{}  % no special notice (required even if empty)
% Or, if applicable, use the standard equal contribution text:
% \printAffiliationsAndNotice{\icmlEqualContribution}

\begin{abstract}
We study sequential expert routing in non-stationary time series with censored (bandit-style) feedback and time-varying expert availability:
at each round, the router observes the target but only the queried expert's prediction.
We model signed expert residuals with a factorized switching linear-Gaussian state-space model comprising a context-dependent regime
process, a shared global factor, and per-expert idiosyncratic states.
To scale inference to large and evolving expert registries, we derive an IMM-style filter with factorized updates that maintains
per-expert marginals, supports expert entry and pruning, and jointly updates only the queried expert and the shared factor.
Using one-step-ahead predictive beliefs, we apply an information-directed routing rule that trades off predicted cost against
information gain about the latent regime and shared factor. We show experimentally that our framework outperforms
both contextual bandits and adapted offline learning-to-defer methods.
\end{abstract}


\section{Introduction}

Learning-to-defer (L2D) studies decision systems that \emph{route} each query to one of several experts and incur expert-dependent
\emph{consultation costs} \citep{madras2018predict,mozannar2021consistent,Narasimhan,mao2023twostage, montreuil2025ask}.
Most L2D work is studied in an \emph{offline} regime: a routing policy is learned from a fixed dataset, typically under i.i.d.\ assumptions,
and training often relies on supervision that is unavailable online, such as access to \emph{all} experts' predictions or losses for the same input.


In sequential problems, decisions and observations are interleaved over time and the offline assumptions above become
impractical.
At round $t$, the router observes a context $\mathbf{x}_t$ and a set of available experts $\mathcal{E}_t$, selects an expert $I_t\in\mathcal{E}_t$,
and then observes the target $\vy_t$ together with the queried prediction $\vhaty_{t,I_t}$.
Feedback is \emph{censored}: the predictions of unqueried experts remain unobserved.
Moreover, the stream is \emph{non-i.i.d.} and often non-stationary \citep{hamilton2020time, sezer2020financial}, so expert capability
and cross-expert dependence can drift or switch regimes over time.
The expert pool can also change, with experts becoming unavailable or newly arriving, and in operational settings experts may be scarce
resources that must be allocated under availability constraints.
These features make a direct transfer of offline L2D formulations insufficient and motivate online methods that explicitly reason
over time, uncertainty, and resource constraints.


To address these challenges, we develop a probabilistic routing framework for non-stationary time series
under censored feedback and a dynamic expert pool.
We model expert residuals with a switching linear-Gaussian state-space
\citep{ghahramani2000variational, linderman2016recurrent, hu2024modeling} model
that couples a shared global factor with
expert-specific idiosyncratic states and a discrete regime process, enabling time-varying cross-expert dependence.
Faithful to practical settings, we support adding or removing experts without affecting the maintained marginals of retained experts.
We also propose an exploration rule based on the IDS framework \citep{russo2014learning} that trades off predictive cost and information gain
about latent states and regimes.


%
%\paragraph{Contributions.}
%Our main contributions are:
%\begin{itemize}
%    \item A sequential expert-routing formulation for non-stationary time series with censored feedback and a time-varying expert pool (Section~\ref{sec:problem_formulation}).
%    \item A factorized switching state-space model for expert residuals with context-dependent regime switching and a shared latent factor enabling cross-expert information transfer (Section~\ref{sec:generative_model}).
%    \item A scalable IMM-style filtering recursion with dynamic registry management that updates only the queried expert jointly with the shared factor, while keeping per-expert marginals (Sections~\ref{sec:generative_model} and~\ref{sec:registry}).
%    \item An information-directed routing rule based on mutual information about $(z_t,\mathbf{g}_t)$, together with a predictive scheduling layer that outputs on-call active sets under availability constraints (Sections~\ref{sec:exploration} and~\ref{sec:scheduling}).
%\end{itemize}

\section{Related Work}
L2D extends selective prediction \citep{Chow_1970, Bartlett_Wegkamp_2008, cortes, Geifman_El-Yaniv_2017, cao2022generalizing, cortes2024cardinalityaware}
by allowing a learner to defer uncertain inputs to external experts \citep{madras2018predict, mozannar2021consistent, Verma2022LearningTD}.
An important line of work develops surrogate losses and statistical guarantees  \citep{mozannar2021consistent, Verma2022LearningTD,
    Cao_Mozannar_Feng_Wei_An_2023, Mozannar2023WhoSP, mao2024realizablehconsistentbayesconsistentloss,
    mao2025realizablehconsistentbayesconsistentloss, charusaie2022sample,
    mao2024principledapproacheslearningdefer, wei2024exploiting}.
L2D has also been extended to regression and multi-task settings and applied in real systems
\citep{mao2024regressionmultiexpertdeferral,
    strong2024towards, palomba2025a, montreuil2024twostagelearningtodefermultitasklearning, montreuil2025optimalqueryallocationextractive}.
Missing expert predictions have been studied in offline/batch learning \citep{nguyen2025probabilistic}.
Sequential L2D has been studied in a different setting: \citet{joshi2021learning} formulate deferral in a non-stationary MDP and learn a \emph{pre-emptive} deferral policy by comparing the long-term value of deferring now versus delaying deferral.

In contrast, we study time-series expert routing where the router selects among available experts \emph{online} under censored (bandit-style) feedback,
with potentially non-stationary data and a time-varying expert pool.
We are not aware of existing L2D formulations that jointly address non-stationarity, censored observations, and dynamic expert availability.




\section{Background}
\label{sec:background}
\subsection{Offline Learning-to-Defer}
\label{sec:background_l2d}
We briefly recall the standard \emph{offline} learning-to-defer (L2D) setup
\citep{madras2018predict, mozannar2021consistent, Narasimhan, mao2024regressionmultiexpertdeferral}.
In its simplest form, one observes i.i.d.\ samples \((\mathbf{x},\vy)\sim\mathcal{D}\), where
\(\mathbf{x}\in\mathcal{X}\subseteq\mathbb{R}^{d}\) and \(\vy\in\mathcal{Y}\subseteq\mathbb{R}^{d_y}\).
There is a fixed registry \(\mathcal{K}=\{1,\dots,K\}\) of experts (or predictors), each providing a prediction
\(\vhaty_k(\mathbf{x})\in\mathcal{Y}\) when queried.  Given a per-expert consultation fee \(\beta_k\ge 0\) and a loss on the prediction error
\(\psi:\mathbb{R}^{d_y}\to\mathbb{R}_{\ge 0}\), the incurred cost of routing \((\mathbf{x},\vy)\) to
expert \(k\) is
\begin{equation}
\label{eq:l2d_cost}
C_k(\mathbf{x},\vy)\coloneqq \psi\big(\vhaty_k(\mathbf{x})-\vy\big)+\beta_k.
\end{equation}
A router is a policy \(\pi:\mathcal{X}\to\Delta^{K-1}\) mapping each input to a
distribution over experts. Its population objective is the expected routing cost
\begin{equation}
\label{eq:l2d_objective}
\mathcal{R}(\pi)
\coloneqq
\mathbb{E}_{(\mathbf{x},\vy)\sim\mathcal{D}}\left[\sum_{k=1}^{K}\pi(k\mid \mathbf{x})\,C_k(\mathbf{x},\vy)\right].
\end{equation}
Conditioned on \(\mathbf{x}\), the Bayes-optimal deterministic router selects
\begin{equation}
\label{eq:l2d_bayes_rule}
k^\star(\mathbf{x})
\in
\arg\min_{k\in\mathcal{K}} \ \mathbb{E}\left[C_k(\mathbf{x},\vy)\mid \mathbf{x}\right].
\end{equation}

If the router selects expert \(I\in\mathcal{K}\) on input \(\mathbf{x}\) with outcome \(\vy\), the incurred cost is \(C_I(\mathbf{x},\vy)\).
Thus, conditioned on \(\mathbf{x}\), the Bayes-optimal deterministic router chooses the expert with the smallest conditional expected cost, as in \eqref{eq:l2d_bayes_rule}.

Most prior works learn \(\pi\) from a fixed dataset by empirical risk minimization on a dedicated surrogate loss \citep{mozannar2021consistent}, often
assuming access to all experts' predictions \((\vhaty_k(\mathbf{x}_i))_{k=1}^K\) (or equivalently all
costs \((C_k(\mathbf{x}_i,\vy_i))_{k=1}^K\)) for each training sample. Practical algorithms
parameterize \(\pi\) with a model (e.g., a neural network) and may use surrogates or relaxations to
handle discrete routing decisions and to obtain statistical guarantees
\citep{mozannar2021consistent, Verma2022LearningTD, mao2024regressionmultiexpertdeferral}.

\subsection{Non-Stationary Time Series and SSMs}
\label{sec:background_slds}

The offline L2D formulation above assumes i.i.d.\ data under a fixed distribution \(\mathcal{D}\).
In time-series, the data-generating process is typically \emph{non-stationary}: the joint law of a process need not be
invariant to time shifts \citep{hamilton2020time}.
In many learning problems with observed contexts, this manifests as \emph{time-varying conditional
laws} (concept drift), i.e., the conditional distribution of \(\vy_t\) given \(\mathbf{x}_t\) can
evolve with \(t\).

State-space models (SSMs) provide a standard probabilistic representation of such non-stationarity by
introducing a latent state \(\mathbf{h}_t\) capturing time-varying conditions \citep{rabiner2003introduction, shumway2006time}.
In our setting, the observation will later correspond to an expert residual.
In a linear-Gaussian SSM,
\begin{align}
    \mathbf{h}_t &= A \mathbf{h}_{t-1} + w_t,\qquad w_t\sim \mathcal{N}(0,Q),\\
    r_t &= C \mathbf{h}_t + v_t,\qquad v_t\sim \mathcal{N}(0,R),
\end{align}
and the Kalman filter \citep{kalman1960new, welch1995introduction} yields tractable online posteriors and predictive
uncertainties.
Switching linear dynamical systems (SLDSs) \citep{bengio1994input, ghahramani2000variational, fox2008nonparametric, hu2024modeling, geadah2024parsing} enrich this model with a discrete regime variable
\(z_t\in\{1,\dots,M\}\) selecting among multiple linear-Gaussian dynamics; conditioned on \(z_t=m\),
\((A,Q,C,R)\) are replaced by \((A_m,Q_m,C_m,R_m)\).

\section{Context-Aware Routing in Non-Stationary Environments}

\subsection{Problem Formulation}
\label{sec:problem_formulation}

Building on the offline learning-to-defer setup in Section~\ref{sec:background_l2d}, we study
\emph{sequential} expert routing in \emph{non-stationary} time series under \emph{censored
feedback} \citep{neu2010online, dani2008stochastic}.

\textbf{Primitives.}
Time is indexed by a finite horizon \(t\in[T]\coloneqq\{1,\dots,T\}\). Let
\((\Omega,\mathcal{F},\mathbb{P})\) be a probability space supporting all random variables. At each
round \(t\), the environment produces a context \(\mathbf{x}_t\in\mathcal{X}\subseteq\mathbb{R}^d\), a
target \(\vy_t\in\mathcal{Y}\subseteq\mathbb{R}^{d_y}\) with \(d_y\ge 1\), and a non-empty finite set
of available expert identities \(\mathcal{E}_t\). We allow \(\mathcal{E}_t\) to vary with \(t\),
capturing both temporary unavailability and newly arriving experts.
The router maintains a time-varying \emph{expert registry} \(\mathcal{K}_t\), containing the experts
for which it stores per-expert state, with \(\mathcal{E}_t\subseteq \mathcal{K}_t\) at decision time.
For scalability, \(\mathcal{K}_t\) may discard stale experts and reinitialize them upon re-entry
(details in Section~\ref{sec:registry}).
Each identity \(k\in\mathcal{K}_t\) corresponds to a persistent expert that, when queried at time \(t\),
outputs a prediction \(\vhaty_{t,k}\in\mathcal{Y}\).

\textbf{Residuals, loss, and cost.}
As in \eqref{eq:l2d_cost}, routing to expert \(k\) incurs a prediction error loss plus a query fee.
We track experts via their signed residuals (prediction minus target). We define the \emph{potential}
residual of expert \(k\) at time \(t\) as
\begin{equation}
\label{eq:residual}
  e_{t,k} \coloneqq \vhaty_{t,k}-\vy_t .
\end{equation}
When \(I_t=k\) is queried, the realized observation is \(e_t\coloneqq e_{t,I_t}\).
We model residuals (rather than the nonnegative loss \(\psi(e_{t,k})\)) because the state-space
emission model is defined on \(\mathbb{R}^{d_y}\), preserving signed deviations (over- vs.\ under-prediction)
that would be lost after applying \(\psi\).
The corresponding (potential) routing cost is
\begin{equation}
\label{eq:routing_cost}
  C_{t,k} \coloneqq \psi(e_{t,k}) + \beta_k .
\end{equation}
where \(\psi:\mathbb{R}^{d_y}\to\mathbb{R}_{\ge 0}\) is a known convex loss (e.g., squared error for
\(d_y=1\) or squared norm \(\psi(e)=\lVert e\rVert_2^2\) in general) and \(\beta_k\ge 0\) is a known,
expert-specific query fee. When \(I_t=k\) is queried, the realized cost is \(C_t\coloneqq C_{t,I_t}\).

\textbf{Observation model (censoring).}
At each round, the router selects an expert index \(I_t\in\mathcal{E}_t\). Due to bandit-style
feedback, it observes only the queried prediction \(\vhaty_{t,I_t}\) (and hence only
\(e_{t,I_t}\) and \(C_{t,I_t}\)); for \(k\in\mathcal{E}_t\setminus\{I_t\}\),
\((\vhaty_{t,k},e_{t,k},C_{t,k})\) remain
unobserved. We denote the post-action feedback tuple by $O_t \coloneqq (I_t,\vhaty_{t,I_t},\vy_t)$.

\textbf{Filtrations and policies.}
Let \(\mathcal{H}_{t}\coloneqq \big((\mathbf{x}_\tau,\mathcal{E}_\tau,O_\tau)\big)_{\tau=1}^{t}\)
be the interaction history up to the end of round \(t\).
Decisions are non-anticipative, i.e., made before observing \(O_t\).
We define the \emph{decision-time} sigma-algebra as $\mathcal{F}_t \coloneqq
\sigma\left(\mathcal{H}_{t-1},\mathbf{x}_t,\mathcal{E}_t\right)$.

A policy \(\pi=(\pi_t)_{t=1}^T\) is a sequence of decision rules where
\(\pi_t(\cdot \mid \mathcal{F}_t)\) is an \(\mathcal{F}_t\)-measurable distribution over
\(\mathcal{E}_t\).
The action is sampled as \(I_t \sim \pi_t(\cdot \mid \mathcal{F}_t)\), so that
\(I_t\in\mathcal{E}_t\) almost surely.

\textbf{Interaction protocol.}
The process unfolds in discrete rounds. At each time \(t\):
\begin{enumerate}[topsep=2pt,itemsep=1pt,parsep=0pt]
    \item \textbf{Decision-time revelation:} the environment reveals \((\mathbf{x}_t,\mathcal{E}_t)\),
    thereby determining \(\mathcal{F}_t\).
    \item \textbf{Action:} the router samples \(I_t \sim \pi_t(\cdot \mid \mathcal{F}_t)\).
    \item \textbf{Censored feedback:} the router observes \(O_t=(I_t,\vhaty_{t,I_t},\vy_t)\) and can
    evaluate the realized residual \(e_{t,I_t}\) and cost \(C_{t,I_t}\).
\end{enumerate}


\textbf{Non-stationarity and exogeneity.}
We do not assume i.i.d.\ data: the joint law of \((\mathbf{x}_t,\mathcal{E}_t,\vy_t)\) may drift over
time (Section~\ref{sec:background_slds}). Concretely, we allow a sequence of time-varying conditional
laws \(\{\mathcal{D}_t\}_{t\ge 1}\) such that
	\begin{equation*}
	    (\mathbf{x}_t,\mathcal{E}_t,\vy_t)\mid \sigma\big((\mathbf{x}_\tau,\mathcal{E}_\tau,\vy_\tau)_{\tau<t}\big)
	    \sim \mathcal{D}_t\left(\cdot \middle| (\mathbf{x}_\tau,\mathcal{E}_\tau,\vy_\tau)_{\tau<t}\right).
	\end{equation*}
This captures non-stationarity (e.g., concept drift or regime shifts). We additionally assume
\emph{exogeneity}: past routing actions affect which expert predictions
are observed, but do not influence the data-generating process. Equivalently,
\((\mathbf{x}_t,\mathcal{E}_t,\vy_t)\) is conditionally independent of past actions \(I_{1:t-1}\)
given \(\sigma\big((\mathbf{x}_\tau,\mathcal{E}_\tau,\vy_\tau)_{\tau<t}\big)\).

\textbf{Objective and myopic Bayes selector.}
Our goal is to minimize expected cumulative routing cost
\begin{equation}
\label{eq:routing_objective}
    J(\pi) \coloneqq \mathbb{E}\left[\sum_{t=1}^T C_{t,I_t}\right].
\end{equation}
As an idealized one-step benchmark, the \emph{myopic Bayes selector} chooses
\begin{equation}
\label{eq:bayes_selection}
    k_t^{\star} \in \arg\min_{k\in\mathcal{E}_t} \mathbb{E}\left[C_{t,k} \mid \mathcal{F}_t\right].
\end{equation}
Under full feedback, \eqref{eq:bayes_selection} is directly evaluable from contemporaneous observations
of all experts' costs. Under censoring, however, \(C_{t,k}\) is observed only for the queried expert
\citep{neu2010online}, so \eqref{eq:bayes_selection} is not directly computable.
Since \(\beta_k\) is known, evaluating \eqref{eq:bayes_selection} reduces to forecasting
\(\mathbb{E}[\psi(e_{t,k})\mid \mathcal{F}_t]\) for unqueried experts. In subsequent sections, we introduce a latent-state model that yields tractable
one-step-ahead predictive beliefs \(p(e_{t,k}\mid \mathcal{F}_t)\).

\subsection{Generative Model: Factorized Switching LDS}
\label{sec:generative_model}

Section~\ref{sec:problem_formulation} highlights that censored feedback and non-stationarity make the
myopic selector \eqref{eq:bayes_selection} intractable without a predictive belief over \emph{unobserved}
expert residuals.

We therefore model the \emph{potential residuals} \(e_{t,k}=\vhaty_{t,k}-\vy_t\) from
Section~\ref{sec:problem_formulation} as emissions of a \textbf{factorized switching linear dynamical system
} \citep{bengio1994input, linderman2016recurrent, hu2024modeling}. The central bottleneck is censoring: at round \(t\) we observe only the queried residual
\(e_t\coloneqq e_{t,I_t}\), while \((e_{t,k})_{k\neq I_t}\) remain counterfactual. We
address this by combining (i) a \emph{switching} latent regime \(z_t\) to capture abrupt changes, (ii)
a \emph{shared} global factor \(\mathbf{g}_t\) that couples experts and enables information transfer,
and (iii) \emph{idiosyncratic} expert-specific dynamics \(\mathbf{u}_{t,k}\). For scalability under a
growing registry, our inference later maintains per-expert marginals via a factorized filtering
approximation. The resulting linear-Gaussian structure yields Kalman-style updates and closed-form
information quantities used in our routing rule.

\subsubsection{Latent state hierarchy}
We represent non-stationarity via a two-level hierarchy separating systemic shifts from
expert-specific drifts.
The hierarchy is designed so that a single queried residual can update a \emph{shared} latent factor
\(\mathbf{g}_t\), which immediately refines predictions for all experts. Expert-specific states
\(\mathbf{u}_{t,k}\) then capture persistent idiosyncratic deviations that cannot be explained by global
conditions alone.

\textbf{Context-dependent regime switching.}
A discrete regime \(z_t\in\{1,\dots,M\}\) selects the active dynamical law (e.g., ``bull'' vs.\ ``crisis'').
While classical SLDSs often use a time-homogeneous transition matrix, we allow transition
probabilities to depend on the observed context \(\mathbf{x}_t\) (input-driven switching; e.g.,
\citet{bengio1994input}). Let
\(\Pi_\theta(\mathbf{x}_t)\in[0,1]^{M\times M}\) be a row-stochastic matrix with
\begin{equation*}
    \mathbb{P}(z_t=m \mid z_{t-1}=\ell,\mathbf{x}_t)=\Pi_\theta(\mathbf{x}_t)_{\ell m}.
\end{equation*}
This lets the filter incorporate exogenous signals in \(\mathbf{x}_t\) to update its regime belief
before observing the queried residual \(e_t\). Contexts that shift mass toward regime \(m\) favor
experts with low mode-\(m\) predicted cost, yielding
an interpretable link between \(\mathbf{x}_t\), regimes, and expert specialization.

We parameterize the logits of \(\Pi_\theta(\mathbf{x}_t)\) with a low-rank scaled-attention form to
control statistical and computational complexity \citep{allyouneed, kossen2021self, mehta2022neural}.
Specifically, for a chosen bottleneck dimension \(d_{\mathrm{attn}}\), we compute
\(Q_\theta(\mathbf{x}_t),K_\theta(\mathbf{x}_t)\in\mathbb{R}^{M\times d_{\mathrm{attn}}}\) and set
\[
S(\mathbf{x}_t)\coloneqq \frac{1}{\sqrt{d_{\mathrm{attn}}}}Q_\theta(\mathbf{x}_t)K_\theta(\mathbf{x}_t)^\top,
\]
so that \(\mathrm{rank}(S(\mathbf{x}_t))\le d_{\mathrm{attn}}\). Applying a row-wise softmax yields the
transition matrix:
\begin{equation}
\label{eq:context_transitions}
\mathbb{P}(z_t=m \mid z_{t-1}=\ell,\mathbf{x}_t)
=
\frac{\exp(S_{\ell m}(\mathbf{x}_t))}{\sum_{j=1}^M \exp(S_{\ell j}(\mathbf{x}_t))}.
\end{equation}

\textbf{Global factor dynamics.}
Under censored feedback, the only way to learn about \emph{unqueried} experts is to exploit structure
that couples them (see Proposition \ref{prop:cross_update}). We therefore introduce a continuous \emph{shared} latent state
\(\mathbf{g}_t\in\mathbb{R}^{d_g}\) representing system-wide conditions (e.g., overall difficulty,
market volatility, sensor drift) that affect many experts simultaneously. Because \(\mathbf{g}_t\)
appears in every expert's residual model, updating \(\mathbf{g}_t\) from the single observed residual
\(e_t=e_{t,I_t}\) tightens the predictive beliefs for other experts \(k\neq I_t\),
providing the cross-expert information transfer needed for routing.

Conditioned on \(z_t=m\), we model \(\mathbf{g}_t\) with linear-Gaussian dynamics to retain Kalman-style
updates and closed-form predictive quantities used later for exploration:
\begin{equation}
\label{eq:global_dynamics}
\mathbf{g}_t
=
\mathbf{A}^{(g)}_{m}\mathbf{g}_{t-1}+\mathbf{w}^{(g)}_{t},
\qquad
\mathbf{w}^{(g)}_{t}\sim\mathcal{N}(\mathbf{0},\mathbf{Q}^{(g)}_{m}),
\end{equation}
where \(\mathbf{A}^{(g)}_{m}\in\mathbb{R}^{d_g\times d_g}\) and
\(\mathbf{Q}^{(g)}_{m}\in\mathbb{S}^{d_g}_{++}\).
We assume \((\mathbf{w}^{(g)}_{t})_{t}\) are independent across time and independent of all other
process and emission noise terms.

\textbf{Expert-specific dynamics.}
Not all variation is shared: experts can drift due to recalibration, local overfitting, model
updates, or intermittent failures. We capture these \emph{idiosyncratic} effects with a per-expert
latent state \(\mathbf{u}_{t,k}\in\mathbb{R}^{d_\alpha}\). Conditioned on \(z_t=m\),
\begin{equation}
\label{eq:idiosyncratic_dynamics}
\mathbf{u}_{t,k}
=
\mathbf{A}^{(u)}_{m}\mathbf{u}_{t-1,k}+\mathbf{w}^{(u)}_{t,k},
\quad
\mathbf{w}^{(u)}_{t,k}\sim\mathcal{N}(\mathbf{0},\mathbf{Q}^{(u)}_{m}),
\end{equation}
where conditional on \((z_t)\), the noise terms are independent across experts and time.
To maintain statistical strength under sparse feedback, we share the dynamics parameters
\((\mathbf{A}^{(u)}_{m},\mathbf{Q}^{(u)}_{m})\) across experts.

\textbf{Reliability composition and residual emission.}
Expert heterogeneity is then expressed
through (i) the expert-specific state realization \(\mathbf{u}_{t,k}\) and (ii) expert-specific
loadings \(\mathbf{B}_k\), which determine how each expert responds to the
shared factor \(\mathbf{g}_t\).

\begin{definition}[L2D-SLDS reliability and residual emission]
\label{def:l2d_slds_emission}
	Fix latent dimensions \(d_g\) and \(d_\alpha\) and a feature map
	\(\Phi:\mathcal{X}\to\mathbb{R}^{d_\alpha\times d_y}\).
	For each expert \(k\), define its latent \emph{reliability} vector at time \(t\) by
	\begin{equation}
	\label{eq:alpha_def}
	\boldsymbol\alpha_{t,k}\coloneqq \mathbf{B}_k\mathbf{g}_t+\mathbf{u}_{t,k},
	\qquad
	\mathbf{B}_k\in\mathbb{R}^{d_\alpha\times d_g}.
	\end{equation}
	Given regime \(z_t=m\), context \(\mathbf{x}_t\), and latent states \((\mathbf{g}_t,\mathbf{u}_{t,k})\),
	the signed residual \(e_{t,k}=\vhaty_{t,k}-\vy_t\) is generated by the linear-Gaussian emission
	\begin{equation}
	\label{eq:residual_emission}
	e_{t,k}\mid (z_t=m,\mathbf{g}_t,\mathbf{u}_{t,k},\mathbf{x}_t)
	\sim
	\mathcal{N}\big(\Phi(\mathbf{x}_t)^\top\boldsymbol\alpha_{t,k},\mathbf{R}_{m,k}\big),
	\end{equation}
	where \(\mathbf{R}_{m,k}\in\mathbb{S}^{d_y}_{++}\) is an expert- and regime-specific noise covariance.
\end{definition}

Definition~\ref{def:l2d_slds_emission} is the \emph{residual emission} component of our L2D-SLDS: it
makes expert performance depend on the observed context via \(\Phi(\mathbf{x}_t)\) while preserving
linear-Gaussian structure (hence Kalman-style updates and closed-form predictive quantities). We
assume emission noise is conditionally independent across experts and time given
\((z_t,\mathbf{g}_t,(\mathbf{u}_{t,k})_k)\). We assume an initial distribution \(p(z_1)\) and Gaussian priors for \(\mathbf{g}_0\) and
\(\mathbf{u}_{0,k}\); inference only requires these to be specified and known.

\subsubsection{Implications of the Hierarchy}

\textbf{Selective information transfer via factorization.}
The hierarchy is constructed so that routing can generalize across experts through the shared factor
\(\mathbf{g}_t\), while \(\mathbf{u}_{t,k}\) captures persistent expert-specific drift. In the exact
Bayesian filter, conditioning on the single observed residual \(e_t=e_{t,I_t}\) couples
\(\mathbf{g}_t\) with \((\mathbf{u}_{t,k})_k\), and hence couples experts with each other; maintaining
the full joint posterior becomes prohibitive as the registry grows.

For scalability, our inference maintains a \emph{factorized} filtering approximation: after each
update, we project the belief onto a family in which (conditional on \(z_t\)) the idiosyncratic
states are independent across experts and independent of \(\mathbf{g}_t\); see
Appendix~\ref{app:cross_covariance} for the corresponding non-factorized update. This projection
discards posterior cross-covariances, but preserves the mechanism needed under censoring: querying a
single expert updates \(\mathbf{g}_t\), which shifts the predictive residual distributions of \emph{all}
experts through \(\mathbf{B}_k\). The proposition below makes the resulting information transfer
criterion explicit.

\begin{restatable}[Information transfer under a shared factor]{proposition}{propinfo}
\label{prop:cross_update}
Fix \(t\) and \(z_t=m\), and let \(\mathcal{G}_t\coloneqq \sigma(\mathcal{F}_t,I_t,z_t=m)\). Let
\(j\neq I_t\) and let \((e_{t,j}^{\mathrm{pred}},e_{t,I_t}^{\mathrm{pred}})\) denote the one-step-ahead
predictive residuals under \(p(e_{t,\cdot}\mid \mathcal{F}_t,z_t=m)\). Assume that this predictive pair
is jointly Gaussian conditional on \(\mathcal{G}_t\) and that
\(\mathrm{Cov}(e_{t,I_t}^{\mathrm{pred}}\mid \mathcal{G}_t)\) is non-singular (e.g.,
\(\mathbf{R}_{m,I_t}\succ \mathbf{0}\)). Then
\begin{equation*}
    \begin{aligned}
        & \mathbb{E}\left[e_{t,j}^{\mathrm{pred}}\mid e_t,\mathcal{G}_t\right]
=
\mathbb{E}\left[e_{t,j}^{\mathrm{pred}}\mid \mathcal{G}_t\right] \\
& \quad\Longleftrightarrow\quad
\mathrm{Cov}\left(e_{t,j}^{\mathrm{pred}},e_{t,I_t}^{\mathrm{pred}}\mid \mathcal{G}_t\right)=\mathbf{0}.
    \end{aligned}
\end{equation*}
In particular, if the covariance is non-zero, then observing \(e_t=e_{t,I_t}\) updates the posterior
predictive mean of \(e_{t,j}^{\mathrm{pred}}\).
\end{restatable}

We prove Proposition~\ref{prop:cross_update} in Appendix~\ref{app:proof_cross_update}. Observing the queried
residual affects unqueried experts exactly when their predictive
residuals are correlated. In our factorized SLDS, this correlation is induced by the shared factor
\(\mathbf{g}_t\). Under the linear-Gaussian model, the predictive residuals are jointly Gaussian, and
their cross-covariance can be read directly from the shared-factor channel. For example, conditional on
\((\mathcal{F}_t,z_t=m)\),
\(\mathrm{Cov}(e_{t,j}^{\mathrm{pred}},e_{t,i}^{\mathrm{pred}})\) contains the shared-factor term
\[
\Phi(\mathbf{x}_t)^\top \mathbf{B}_j \Sigma^{(m)}_{g,t\mid t-1}\mathbf{B}_{i}^\top \Phi(\mathbf{x}_t),
\]
where \(\Sigma^{(m)}_{g,t\mid t-1}\) is the one-step predictive covariance of \(\mathbf{g}_t\)
under regime \(m\). Thus, querying \(i=I_t\) tightens expert \(j\)'s predictive distribution whenever
the coupling through \(\mathbf{g}_t\) is non-negligible in the directions probed by \(\Phi(\mathbf{x}_t)\).
Conversely, if this term vanishes, then under the
factorized predictive belief there is no information transfer from \(I_t\) to \(j\) at time \(t\).

\subsubsection{Exploration via Information-Directed Sampling}
\label{sec:exploration}

Under censored feedback, greedily selecting the expert with the lowest predicted cost can slow
adaptation by repeatedly querying a ``safe'' expert. We therefore use
\textit{Information-Directed Sampling (IDS)} \citep{russo2014learning} to trade off predicted cost against
information about the
latent state \((z_t,\mathbf{g}_t)\).

\textbf{Exploitation: predicted cost and gap.}
For each \(k\in\mathcal{E}_t\), the model provides a one-step-ahead predictive residual
\(e_{t,k}^{\mathrm{pred}}\sim p(e_{t,k}\mid\mathcal{F}_t)\) and predicted cost
\[
\bar C_{t,k}^{\mathrm{pred}}
\coloneqq
\mathbb{E}\!\left[\psi(e_{t,k}^{\mathrm{pred}})\,\middle|\,\mathcal{F}_t\right]+\beta_k.
\]
Let \(k_t^{\mathrm{pred}} \in \arg\min_{k\in\mathcal{E}_t} \bar C_{t,k}^{\mathrm{pred}}\) be the myopic
predictor. We define the predictive value gap
\begin{equation}
\label{eq:model_gap}
\Delta_t(k)
\coloneqq
\bar C_{t,k}^{\mathrm{pred}}-\bar C_{t,k_t^{\mathrm{pred}}}^{\mathrm{pred}}
\ge 0 .
\end{equation}


\textbf{Exploration: informativeness of a query.}
We quantify the informativeness of querying \(k\) by the mutual information between the latent state
and the (hypothetical) queried residual:
\begin{equation}
\label{eq:ig_operational}
\mathrm{IG}_t(k)
\coloneqq
\mathcal{I}\left((z_t,\mathbf{g}_t); e_{t,k}^{\mathrm{pred}} \middle| \mathcal{F}_t\right).
\end{equation}
For our model, the shared-factor component admits a closed form, while the regime-identification
component is estimated with a lightweight Monte Carlo routine; see Remark~\ref{rmk:zg_information}
(Appendix) and Algorithm~\ref{alg:router_main}.

\textbf{Minimizing the information ratio.}
IDS selects the routing action by minimizing the squared information ratio
\begin{equation}
\label{eq:ids_deterministic}
I_t \in \arg\min_{k\in\mathcal{E}_t}\ \frac{\Delta_t(k)^2}{\mathrm{IG}_t(k)}.
\end{equation}
We interpret the ratio as \(+\infty\) when \(\mathrm{IG}_t(k)=0\) unless \(\Delta_t(k)=0\); if all
\(\mathrm{IG}_t(k)=0\), IDS reduces to the myopic choice \(k_t^{\mathrm{pred}}\).


\subsubsection{Dynamic Registry Management}
\label{sec:registry}

In many deployments, expert availability varies and the pool evolves over time. A static
learning-to-defer router \citep{madras2018predict, mozannar2021consistent} trained on a fixed expert catalog does not naturally support
adding experts
without retraining, nor dropping expert-specific components to reclaim memory/compute.

Our state-space approach makes this issue explicit: each expert \(k\) carries an idiosyncratic latent
state \(\mathbf{u}_{t,k}\) that must be stored and propagated for prediction. When the pool is large
or long-lived, we cannot maintain \(\mathbf{u}_{t,k}\) for every expert ever encountered. We therefore
treat expert-specific state as a \emph{cache} and manage it online.

Recall that \(\mathcal{K}_t\) denotes the router's maintained expert registry
(Section~\ref{sec:problem_formulation}): experts for which we store per-expert filtering marginals,
i.e., maintain \(\mathbf{u}_{t,k}\). The registry is not cumulative: experts may be removed when stale
and re-added upon re-entry, while maintaining \(\mathcal{E}_t\subseteq \mathcal{K}_t\) at decision time
and keeping \(|\mathcal{K}_t|\) bounded.

\textbf{Pruning.}
Let \(\tau_{\mathrm{last}}(k)\in\{0,1,\dots,t-1\}\) be the last round at which expert \(k\) was queried
(with the convention \(\tau_{\mathrm{last}}(k)=0\) if \(k\) has never been queried).
We call an expert \emph{stale} if it is currently unavailable and has not been queried for more than
\(\Delta_{\max}\) steps, where \(\Delta_{\max}\ge 1\) is a user-chosen staleness horizon:
\begin{equation}
\label{eq:stale_set}
\mathcal{K}^{\mathrm{stale}}_t
\coloneqq
\left\{k\in \mathcal{K}_{t-1}\setminus \mathcal{E}_t:\ t-\tau_{\mathrm{last}}(k)>\Delta_{\max}\right\}.
\end{equation}
We update the registry by first adding currently available experts and then pruning stale ones:
\begin{equation}
\label{eq:registry_update}
\mathcal{K}_t \coloneqq (\mathcal{K}_{t-1}\cup \mathcal{E}_t)\setminus \mathcal{K}^{\mathrm{stale}}_t,
\qquad \mathcal{K}_0=\varnothing .
\end{equation}
Since \(\mathcal{K}^{\mathrm{stale}}_t\subseteq \mathcal{K}_{t-1}\setminus \mathcal{E}_t\) by
construction, \eqref{eq:registry_update} guarantees \(\mathcal{E}_t\subseteq \mathcal{K}_t\).
Operationally, pruning means we stop storing the idiosyncratic filtering marginal(s) associated with
\(\mathbf{u}_{t-1,k}\) (and hence do not propagate it forward) for \(k\in\mathcal{K}^{\mathrm{stale}}_t\).

Pruning does \emph{not} alter the maintained belief over retained variables: it is exact
marginalization of dropped coordinates in the filtering distribution.

\begin{restatable}[Pruning does not affect retained experts]{proposition}{invariance}
\label{prop:invariance}
Fix time \(t\) and let \(P_t \subseteq \mathcal{K}_{t-1}\) be any set of experts to be pruned.
Let
\(
q_{t-1\mid t-1}\big(\mathbf{g}_{t-1},(\mathbf{u}_{t-1,\ell})_{\ell\in\mathcal{K}_{t-1}}\big)
\)
denote the (exact or approximate) filtering belief at the end of round \(t-1\) conditioned on the realized history.
Define the pruned belief by marginalization:
\begin{equation*}
    \begin{aligned}
        q^{\mathrm{pr}(P_t)}_{t-1\mid t-1}\big(\mathbf{g}_{t-1},(\mathbf{u}_{t-1,\ell})_{\ell\in\mathcal{K}_{t-1}\setminus P_t}\big)
& \coloneqq \\
\int q_{t-1\mid t-1}\big(\mathbf{g}_{t-1},(\mathbf{u}_{t-1,\ell})_{\ell\in\mathcal{K}_{t-1}}\big)
\prod_{k\in P_t} d\mathbf{u}_{t-1,k}.
    \end{aligned}
\end{equation*}
Then \(q^{\mathrm{pr}(P_t)}_{t-1\mid t-1}\) equals the marginal of \(q_{t-1\mid t-1}\) on the retained variables.
Consequently, after applying the standard SLDS time update to obtain the predictive belief at round \(t\),
the predictive distribution of \(\boldsymbol\alpha_{t,\ell}\) and the one-step predictive law of
\(e_{t,\ell}^{\mathrm{pred}}\) are identical before and after pruning, for every retained \(\ell\notin P_t\).
\end{restatable}

We defer the proof to Appendix~\ref{app:invariance}. If a pruned expert later reappears, we treat it
as a re-entry and reinitialize its idiosyncratic state; \(\Delta_{\max}\) controls the resulting
memory--accuracy trade-off.

\textbf{Birth and re-entry.}
Let
\(\mathcal{E}^{\mathrm{init}}_t\coloneqq \mathcal{E}_t\setminus \mathcal{K}_{t-1}\)
denote experts that \emph{enter} the maintained registry at time \(t\) (either newly observed or
re-entering after pruning). For each \(j\in\mathcal{E}^{\mathrm{init}}_t\), the filter must
instantiate an idiosyncratic state \(\mathbf{u}_{t,j}\) before the router can assign a calibrated
predictive belief to \(e_{t,j}\). We do so at the \emph{predictive} time (before observing any
residual at round \(t\)).

For each entering expert \(j\) and each regime \(m\in[M]\), we assume an
initialization prior
\begin{equation}
\label{eq:birth_prior}
\mathbf{u}_{t-1,j}\mid(z_t=m)\sim \mathcal{N}\big(\mu^{(m)}_{\mathrm{init},j},\Sigma^{(m)}_{\mathrm{init},j}\big).
\end{equation}

The parameters \((\mu^{(m)}_{\mathrm{init},j},\Sigma^{(m)}_{\mathrm{init},j})\) can be set from side
information when available, or to a conservative default. On entry, we assume the router is provided
with \(\beta_j\), an emission-noise specification \(\mathbf{R}_{m,j}\) (or a shared \(\mathbf{R}_m\)),
and a loading matrix \(\mathbf B_j\) (or a default initialization), so the expert can immediately
benefit from the shared factor via \(\boldsymbol\alpha_{t,j}=\mathbf B_j\mathbf g_t+\mathbf u_{t,j}\).

\begin{restatable}[Coupling at birth through the shared factor]{proposition}{transfer}
\label{prop:transfer}
Fix time \(t\) and condition on \((\mathcal{F}_t,z_t=m)\). Under the Factorized SLDS one-step predictive
belief (i.e., with \(\mathrm{Cov}(\mathbf{g}_t,\mathbf{u}_{t,k}\mid\cdot)=\mathbf{0}\) and
\(\mathrm{Cov}(\mathbf{u}_{t,i},\mathbf{u}_{t,j}\mid\cdot)=\mathbf{0}\) for \(i\neq j\)), for any experts \(j\neq k\),
\[
\mathrm{Cov}\left(\boldsymbol\alpha_{t,j},\boldsymbol\alpha_{t,k}\mid \mathcal{F}_t,z_t=m\right)
=
\mathbf{B}_j\Sigma^{(m)}_{g,t\mid t-1}\mathbf{B}_k^\top,
\]
where \(\Sigma^{(m)}_{g,t\mid t-1}\) is the regime-\(m\) one-step predictive covariance of \(\mathbf{g}_t\).
In particular, if the joint predictive law is Gaussian and
\(\mathbf{B}_j\Sigma^{(m)}_{g,t\mid t-1}\mathbf{B}_k^\top\neq \mathbf{0}\),
then \(\boldsymbol\alpha_{t,j}\) and \(\boldsymbol\alpha_{t,k}\) are not independent and hence
\(\mathcal{I}(\boldsymbol\alpha_{t,j};\boldsymbol\alpha_{t,k}\mid \mathcal{F}_t,z_t=m)>0\).
\end{restatable}

We give the proof in Appendix~\ref{app:transfer}.

\section{Experiments Details}
\label{sec:experiments-details}

We provide additional details on the experiments of Section~\ref{section:experiments}, including
experimental setup, hyperparameters, and implementation details.

\paragraph{Compared methods.}
We compare our \textbf{L2D-SLDS} router under bandit feedback to the following baselines.
\emph{(i) Ablation:} L2D-SLDS without the shared global factor (set \(d_g=0\)).
\emph{(ii) Contextual bandits:} LinUCB \citep{li2010contextual} and NeuralUCB
\citep{zhou2020neuralcontextualbanditsucbbased} with more details in \ref{subsection:baselines}.

\paragraph{Metric.}
We report the time-averaged cumulative routing cost over horizon \(T\) (Eq.~\eqref{eq:routing_objective}). Concretely, we
compute the estimate
\(
\hat{J}(\pi) \coloneqq \frac{1}{T}\sum_{t=1}^T C_{t,I_t},
\)
where \(C_{t,I_t}\) is the realized cost of deferring to the selected expert at round \(t\). Lower is better.

\subsection{Baselines} \label{subsection:baselines}

\paragraph{Feedback regimes.}
At round \(t\), the router observes \((\mathbf{x}_t,\mathcal{E}_t)\), chooses \(I_t\in\mathcal{E}_t\),
and then observes \((\vhaty_{t,I_t},\vy_t)\), hence the realized residual \(e_t=e_{t,I_t}\) and realized
cost \(C_t=C_{t,I_t}\), where \(C_{t,k}\coloneqq \psi(e_{t,k})+\beta_k\) and
\(e_{t,k}=\vhaty_{t,k}-\vy_t\) (Appendix~\ref{app:notation}).
\emph{Partial feedback} means only \((\vhaty_{t,I_t},\vy_t)\) is observed after acting.

\paragraph{L2D-SLDS and ablation without \(\mathbf{g}_t\).}
Our method is the model-based router of Algorithm~\ref{alg:router_main} under the generative residual
model of Definition~\ref{def:l2d_slds_emission}:
\(
\boldsymbol{\alpha}_{t,k}=\mathbf{B}_k\mathbf{g}_t+\mathbf{u}_{t,k}
\)
and
\(
e_{t,k}\mid(z_t=m,\mathbf{g}_t,\mathbf{u}_{t,k},\mathbf{x}_t)\sim
\mathcal{N}\!\big(\Phi(\mathbf{x}_t)^\top\boldsymbol{\alpha}_{t,k},\mathbf{R}_{m,k}\big)
\)
(\eqref{eq:alpha_def}--\eqref{eq:residual_emission}).
\textbf{L2D-SLDS w/o \(\mathbf{g}_t\)} is the ablation obtained by setting \(d_g=0\) (equivalently
\(\mathbf{B}_k\mathbf{g}_t\equiv \mathbf{0}\) for all \(k\)), so that
\(\boldsymbol{\alpha}_{t,k}=\mathbf{u}_{t,k}\) and the per-expert predictive residuals are conditionally
independent across experts under the factorized belief (no cross-expert transfer through a shared
factor).

\paragraph{Contextual bandits: LinUCB and NeuralUCB (partial and full feedback).}
Both methods operate on the per-round \emph{cost} \(C_{t,k}\) and are implemented as \emph{lower}
confidence bound (LCB) rules since we minimize cost.
Under \emph{full feedback}, the router observes \(\{C_{t,k}\}_{k\in\mathcal{E}_t}\) regardless of which
expert \(I_t\) was selected. Consequently, the usual exploration--exploitation trade-off disappears:
the choice of \(I_t\) does not affect what data is available for learning, so the exploration bonus
can be set to \(0\) (yielding greedy selection) without sacrificing statistical efficiency. We still
state the LCB form below for a unified presentation.

\subparagraph{LinUCB.}
Fix a feature map \(\varphi:\mathbb{R}^d\to\mathbb{R}^p\) (in our experiments, either raw
\(\mathbf{x}_t\) or an RNN embedding).
Assume a linear model for the conditional mean cost of each expert:
\(
\mathbb{E}[C_{t,k}\mid \mathbf{x}_t]\approx \varphi(\mathbf{x}_t)^\top \boldsymbol{\theta}_k
\).
Maintain ridge statistics per expert \(k\), with ridge parameter \(\lambda>0\).
Under \emph{partial feedback}:
\[
\mathbf{V}_{t,k}\coloneqq \lambda \mathbf{I}_p + \sum_{s<t:\ I_s=k} \varphi(\mathbf{x}_s)\varphi(\mathbf{x}_s)^\top,
\qquad
\mathbf{b}_{t,k}\coloneqq \sum_{s<t:\ I_s=k} \varphi(\mathbf{x}_s)C_s,
\qquad
\widehat{\boldsymbol{\theta}}_{t,k}\coloneqq \mathbf{V}_{t,k}^{-1}\mathbf{b}_{t,k}.
\]
where \(C_s=C_{s,I_s}\) is the realized (queried) cost at round \(s\).  At time \(t\), set
\(
\widehat{C}_{t,k}\coloneqq \varphi(\mathbf{x}_t)^\top\widehat{\boldsymbol{\theta}}_{t,k}
\)
and exploration bonus
\(
u_t(k)\coloneqq \alpha_t\sqrt{\varphi(\mathbf{x}_t)^\top\mathbf{V}_{t,k}^{-1}\varphi(\mathbf{x}_t)}.
\)
The decision rule is
\[
I_t \in \arg\min_{k\in\mathcal{E}_t}\ \widehat{C}_{t,k}-u_t(k).
\]
\emph{Partial feedback LinUCB} updates only the chosen arm \(I_t\) (so only \(C_{t,I_t}=C_t\) is observed).

\subparagraph{NeuralUCB.}
Let \(f_{\boldsymbol{\omega}}(\mathbf{x},k)\) be a neural predictor of the conditional mean cost of expert
\(k\) given \(\mathbf{x}\) (we use a shared encoder with a per-expert head).
Define a parameter-gradient feature (to avoid overloading the shared factor \(\mathbf{g}_t\))
\(
\mathbf{h}_{t,k}\coloneqq \nabla_{\boldsymbol{\omega}} f_{\boldsymbol{\omega}}(\mathbf{x}_t,k)\in\mathbb{R}^{p_\omega}.
\)
Maintain a (regularized) Gram matrix. Under \emph{partial feedback}:
\[
\mathbf{A}_t\coloneqq \lambda \mathbf{I}_{p_\omega}
+
\sum_{s<t} \mathbf{h}_{s,I_s}\mathbf{h}_{s,I_s}^\top.
\]

At time \(t\), set
\(
\widehat{C}_{t,k}\coloneqq f_{\boldsymbol{\omega}}(\mathbf{x}_t,k)
\)
and
\(
u_t(k)\coloneqq \alpha_t\sqrt{\mathbf{h}_{t,k}^\top \mathbf{A}_t^{-1}\mathbf{h}_{t,k}}.
\)
The decision rule is
\[
I_t \in \arg\min_{k\in\mathcal{E}_t}\ \widehat{C}_{t,k}-u_t(k).
\]
The network is trained online by stochastic gradient steps on squared error.
\emph{Partial feedback NeuralUCB} uses the loss \((f_{\boldsymbol{\omega}}(\mathbf{x}_t,I_t)-C_t)^2\) (only \(C_t\) observed).

\paragraph{Oracle baseline.}
The (per-round) oracle chooses the best available expert in hindsight:
\[
I_t^{\mathrm{oracle}} \in \arg\min_{k\in\mathcal{E}_t} C_{t,k}.
\]
This is infeasible under partial feedback because \(C_{t,k}\) is not observed for all \(k\), but we
report it as a lower bound on achievable cumulative cost.


\subsection{Synthetic: Regime-Dependent Correlation and Information Transfer}
\label{sec:exp_synthetic_transfer_appendix}

\paragraph{Design goal.}
We construct a controlled routing instance in which (i) experts are \emph{correlated} in a
regime-dependent way, so that observing one expert should update beliefs about others (information
transfer; Proposition~\ref{prop:cross_update}); and (ii) one expert temporarily disappears and
re-enters, so that the maintained registry \(\mathcal{K}_t\) matters (see Appendix).

\paragraph{Environment (regimes, target, context).}
We use \(M=2\) regimes and deterministic switching in blocks of length \(L=150\) over horizon
\(T=3000\) such as $z_t \coloneqq 1 + \left\lfloor \frac{t-1}{L}\right\rfloor \bmod 2$.
The target follows a regime-dependent AR(1), and the context is the one-step lag:
\begin{equation}
\label{eq:exp_tri_cycle_ts_appendix}
y_t = 0.8\,y_{t-1} + d_{z_t} + \eta_t,\qquad \eta_t\sim\mathcal{N}(0,\sigma_y^2).
\end{equation}
We set the router's context to \(x_t\coloneqq y_{t-1}\).
The regime \(z_t\) is latent to the router: the router observes only \(x_t\) (before acting) and the
single queried prediction \(\hat y_{t,I_t}\) (after acting).

\paragraph{Experts and availability.}
We use \(K=4\) experts indexed \(k\in\{0,1,2,3\}\). Expert \(k=1\) is removed from the available set \(\mathcal{E}_t\)
for a contiguous interval \(t\in[2000,2500]\) and then re-enters. Each expert is a one-step forecaster
\(\hat y_{t,k}=f_k(x_t)\) with a shared slope and expert-specific intercept plus noise:
\begin{equation}
\label{eq:exp_synth_expert_rule_appendix}
\hat y_{t,k} \coloneqq 0.8\,y_{t-1} + b_k + \varepsilon_{t,k}.
\end{equation}
We set \((b_0,b_1,b_2,b_3)=(d_1,d_1,d_2,d_2)\), so experts \(\{0,1\}\) are well-calibrated in regime
\(z_t=1\) and experts \(\{2,3\}\) are well-calibrated in regime \(z_t=2\).

To induce \emph{regime-dependent correlation} under bandit feedback, we generate the expert noises as
\[
\varepsilon_{t,k} \coloneqq s_{t,g(k)} + \tilde\varepsilon_{t,k},
\qquad
g(k)\coloneqq 1+\mathbf{1}\{k\in\{2,3\}\},
\]
with independent components \(s_{t,1},s_{t,2},(\tilde\varepsilon_{t,k})_{k}\) and regime-dependent
variances $s_{t,1}\sim\mathcal{N}(0,\sigma_{z_t,1}^2), s_{t,2}\sim\mathcal{N}(0,\sigma_{z_t,2}^2),
\tilde\varepsilon_{t,k}\sim\mathcal{N}(0,\sigma_{\mathrm{id}}^2)$, where \((\sigma_{1,1}^2,\sigma_{1,2}^2)=(\sigma_{\mathrm{hi}}^2,\sigma_{\mathrm{lo}}^2)\) and
\((\sigma_{2,1}^2,\sigma_{2,2}^2)=(\sigma_{\mathrm{lo}}^2,\sigma_{\mathrm{hi}}^2)\) with
\(\sigma_{\mathrm{hi}}^2\gg\sigma_{\mathrm{lo}}^2\). This makes experts \(\{0,1\}\) strongly
correlated in regime \(1\) and experts \(\{2,3\}\) strongly correlated in regime \(2\). We report the MSE of each expert in
Table~\ref{tab:exp_avg_costs}.


\begin{table}[ht]
\centering
\small
\setlength{\tabcolsep}{4pt}
\caption{Averaged cumulative cost \eqref{eq:routing_objective} on experiment (Section~\ref{sec:exp_synthetic_transfer}).
We report mean \(\pm\) standard error across five runs. Lower is better.}
\label{tab:exp_avg_costs_appendix}
\begin{tabular}{@{}lcc@{}}
\toprule
Method & Averaged Cumulative Cost \\
\midrule
\textbf{L2D-SLDS} & \(\mathbf{13.58 \pm 0.07}\)  \\
L2D-SLDS w/o \(\mathbf{g}_t\) & \(14.68 \pm 0.01\)  \\
\midrule
LinUCB & \(22.94 \pm 0.01\)  \\
NeuralUCB & \(21.92 \pm 0.31\)  \\
Random & \(26.13 \pm 0.25\)   \\
Always expert 0 & \(23.07\)  \\
Always expert 1 & \(28.66\)  \\
Always expert 2 & \(23.05\)  \\
Always expert 3 & \(29.36\)  \\
Oracle & \(9.04\)\\
\bottomrule
\end{tabular}
\end{table}

\textbf{Model Configuration.}
We use $M = 2$ regimes with shared factor dimension $d_g = 1$ and idiosyncratic dimension $d_\alpha = 1$.
The staleness horizon for pruning is $\Delta_{\max} = 500$. We simply run a small warmup of 100 steps before
running L2D-SLDS and UCBs.


\paragraph{Correlation recovery.}
Figure~\ref{fig:exp_synthetic_transfer_regime_estimation} compares the regime-0 loss correlation
structure. The ground truth exhibits a clear block structure: experts \(\{0,1\}\)
form one correlated group while experts \(\{2,3\}\) form another.
Under partial feedback, L2D-SLDS is the only method that reliably recovers this clustering from
partial observations, whereas removing the shared factor \(\mathbf{g}_t\) blurs the separation and
inflates cross-group correlations, consistent with losing cross-expert information transfer. In
contrast, LinUCB/NeuralUCB yield near-degenerate correlation estimates (e.g., overly uniform or
unstable patterns), reflecting that purely discriminative bandit updates do not maintain a coherent
joint belief over experts' latent error processes.

\paragraph{Results and Analysis.}
Table~\ref{tab:exp_avg_costs} shows that \textbf{L2D-SLDS} achieves the lowest routing cost under
partial feedback (\(13.58\pm 0.07\)), improving over LinUCB/NeuralUCB by a wide margin and also
outperforming the best fixed expert. Crucially, it also beats the ablation that removes the shared
factor \(\mathbf{g}_t\) (\(14.68\pm 0.01\)), a \(\approx 7.5\%\) reduction, which directly supports
our central claim: under censoring, modeling a \emph{global} latent component enables
\emph{cross-expert information transfer} from a single queried residual (see Proposition \ref{prop:transfer}). Intuitively, \(\mathbf{g}_t\)
captures regime-dependent common shocks that couple experts; thus, querying one expert updates beliefs
about unqueried experts in a way that contextual bandits (which treat arms largely independently) and
independent per-expert dynamics cannot replicate.

In Appendix, we provide additional experiments that probe this regime-dependent setting in more depth, including detailed
analyses of expert pruning and re-entry.


\begin{figure}[h]
    \centering
    \includegraphics[width=1\textwidth]{figures/expert_structure_all}
    \caption{We report the selection frequency of each expert over time as a function of the underlying regime.
The top figure corresponds to the oracle, while the bottom figure shows our approach evaluated against the baselines.
By construction, experts~0 and~1 perform better in regime~1, whereas experts~2 and~3 perform better in regime~2.
Accordingly, a well-adapted router should select experts~0 and~1 more frequently in regime~1 and experts~2 and~3 more frequently in regime~2.
L2D-SLDS (with and without $g_t$) is the only method that captures this structure, closely matching the oracle’s selection behavior.
In contrast, LinUCB and NeuralUCB fail to adapt their selection frequencies to the regimes.}
    \label{fig:expert_structure_all}
\end{figure}

\subsection{ETTh1}
\label{sec:exp_etth1_appendix}

\paragraph{Environment.}
We evaluate L2D-SLDS on the ETTh1 electricity transformer temperature dataset~\cite{haoyietal-informer-2021}, using the oil
temperature (OT) channel as the target \(y_t\). We run the router over the full horizon (\(T=17420\) hourly observations).
Following the synthetic setup, the router uses a one-step lag as context, \(x_t \coloneqq y_{t-1}\) (with \(x_0=0\)).
There is no observed regime annotation for ETTh1; the router observes only \(x_t\) before acting and, after selecting
\(I_t\in\mathcal{E}_t\), it observes the realized outcome \(y_t\) and the single queried prediction \(\hat y_{t,I_t}\) (hence the
queried residual \(e_{t,I_t}\)).

\paragraph{Experts and availability.}
We consider \(K=6\) fixed experts (Table~\ref{tab:experts_etth}). To stress-test dynamic availability and our pruning/re-birth
mechanism, we enforce time-varying expert sets: the strong multi-lag baseline (Expert~4) is available only on the interval
\(t\in[1000,2000]\), while Expert~0 is unavailable on the same interval. This prevents degenerate ``always-pick-the-best''
policies and forces the router to handle both expert arrival/departure (Expert~4) and temporary unavailability with later
return (Expert~0).

\begin{table}[ht]
\centering
\small
\setlength{\tabcolsep}{4pt}
\caption{Configuration of experts for ETTh1.}
\label{tab:experts_etth_appendix}
\begin{tabular}{@{}lll@{}}
\toprule
Index & Base & Modification \\
\midrule
\textbf{0} & AR(1) & small variance \\
\textbf{1} & AR(1) & large variance \\
\textbf{2} & MLP & trained on early 2/3 of data \\
\textbf{3} & MLP & trained on late 2/3 of data \\
\textbf{4} & AR multi-lag baseline & using lags [1, 24, 168] \\
\textbf{5} & Constant &  always predict 0 \\
\bottomrule
\end{tabular}
\end{table}

\textbf{Model Configuration.}
We use $M = 5$ regimes with shared factor dimension $d_g = 2$ and idiosyncratic dimension $d_\alpha = 1$.
The staleness horizon for pruning is $\Delta_{\max} = 250$. The exploration term considers information gain on both global factor $g$ and regime $z$.
Online EM adaptation is enabled with a sliding window of $W = 600$ and updates every 300 steps.


\begin{table}[ht]
\centering
\small
\setlength{\tabcolsep}{4pt}
\caption{Averaged cumulative cost \eqref{eq:routing_objective} on ETTh1 (Section~\ref{sec:exp_etth1}).
We report the mean $\pm$ standard error over five runs; lower is better.
The averaged cumulative cost is computed both over the full time horizon and, for each expert, only over the periods during which that expert is available.
This explains why Expert~4 attains a low cost despite being available for only a short duration.
Consequently, it is expected that baseline methods exhibit higher averaged costs than Expert~4.}
\label{tab:exp_avg_costs_etth_appendix}
\begin{tabular}{@{}lcc@{}}
\toprule
Method & Averaged Cumulative Cost  \\
\midrule
\textbf{L2D-SLDS} & \(\mathbf{0.80 \pm 0.06}\)  \\
L2D-SLDS w/o \(\mathbf{g}_t\) & \(0.93 \pm 0.08\) \\
\midrule
LinUCB & \(0.84 \pm 0.01\)  \\
NeuralUCB & \(1.09 \pm 0.19\)  \\
Random & \(14.51 \pm 0.73\)  \\
Always expert 0 & \(0.81\)  \\
Always expert 1 & \(1.19\) \\
Always expert 2 & \(0.77\) \\
Always expert 3 & \(1.21\) \\
Always expert 4 & \(0.74\)  \\
Always expert 5 & \(166.65\) \\
Oracle & \(0.24 \pm 0.01\)  \\
\bottomrule
\end{tabular}
\end{table}

\paragraph{Results and analysis.}
Table~\ref{tab:exp_avg_costs_etth_appendix} reports the averaged cumulative routing cost. Under \emph{partial feedback}, L2D-SLDS achieves
the lowest cost among adaptive methods that learn online from bandit feedback (\(0.80\pm 0.06\)), improving over LinUCB
(\(0.84\)) and substantially outperforming NeuralUCB (\(1.09\pm 0.19\)). Most importantly, removing the shared factor
\(\mathbf{g}_t\) degrades performance
(\(0.93\pm 0.08\)), a relative increase of \(\approx 15\%\). This gap is consistent with the role of \(\mathbf{g}_t\) under
censoring: ETTh1 exhibits common shocks (e.g., global load/temperature patterns) that affect multiple experts similarly, so a
shared latent component lets a single queried residual update beliefs about \emph{unqueried} experts via the learned
cross-expert dependence.



\subsection{FRED: Treasury Securities at 10-Year Constant Maturity}
\label{subsection_appendix_fred}

\paragraph{Environment.}
We evaluate on the FRED DGS10 series (10-year U.S.\ Treasury constant-maturity yield)~\cite{FRED_DGS10}, using the daily
observations in \texttt{data/FRED\_DGS10.csv} from 1990-01-02 through 2023-12-29 (\(T=8506\)).
The target is \(y_t\coloneqq \text{DGS10}_t\). The router uses a fixed context vector \(x_t\in\mathbb{R}^{10}\) consisting of
yield lags at \(\{1,5,20,60,120,250\}\) days and calendar features for day-of-week and month encoded as sine/cosine pairs.
We z-score normalize each context dimension using the first 2520 observations.
As in all partial-feedback experiments, at each round \(t\) the router observes \((x_t,\mathcal{E}_t)\), chooses \(I_t\), and
then observes \(\hat y_{t,I_t}\) and \(y_t\) (hence the queried residual \(e_{t,I_t}\)).

\paragraph{Experts.}
We use \(K=4\) ridge-regularized linear autoregressive experts (AR) of the form \(\hat y_{t,k}=w_k^\top x_t+b_k\), each trained
offline on a disjoint historical date range and then deployed across the full evaluation horizon. To avoid a single expert
becoming deterministically dominant, we add mild i.i.d.\ Gaussian prediction noise with standard deviation \(0.03\) to each
expert's output. All experts are available at all times (\(\mathcal{E}_t=\{0,1,2,3\}\)).

\begin{table}[ht]
\centering
\small
\setlength{\tabcolsep}{4pt}
\caption{Configuration of experts for the FRED DGS10 experiment.}
\label{tab:experts_fred_appendix}
\begin{tabular}{@{}lll@{}}
\toprule
Index & Model & Training window \\
\midrule
\textbf{0} & AR (linear ridge on \(x_t\)) & 1990-01-02--2000-12-31 \\
\textbf{1} & AR (linear ridge on \(x_t\)) & 2001-01-01--2007-12-31 \\
\textbf{2} & AR (linear ridge on \(x_t\)) & 2008-01-01--2015-12-31 \\
\textbf{3} & AR (linear ridge on \(x_t\)) & 2016-01-01--2023-12-31 \\
\bottomrule
\end{tabular}
\end{table}

\textbf{Model Configuration.}
We use \(M=4\) regimes with shared factor dimension \(d_g=2\) and idiosyncratic dimension \(d_\alpha=10\) (matching the context
dimension). The staleness horizon is \(\Delta_{\max}=4000\), measurement noise is set to \(R=0.01\), and exploration uses the
information gain over both \(\mathbf{g}_t\) and \(z_t\) (mode \texttt{g\_z}). We disable EM adaptation in this experiment.

\paragraph{Results and analysis.}
Table~\ref{tab:exp_avg_var_l2d_fred} reports the averaged cumulative routing cost. Under partial feedback, \textbf{L2D-SLDS}
achieves the lowest cost among adaptive methods (\(0.004327\pm0.000003\)), improving over LinUCB, NeuralUCB, and random routing.
Removing the shared factor slightly degrades performance (\(0.004411\pm0.000011\)), consistent with shared latent structure
providing additional cross-expert signal when only one residual is observed per round.

\begin{table}[ht]
\centering
\small
\setlength{\tabcolsep}{4pt}
\caption{Averaged cumulative cost \eqref{eq:routing_objective} on the FRED (DGS10) experiment (Appendix~\ref{subsection_appendix_fred}).
We report mean \(\pm\) standard error across five runs; lower is better.}
\label{tab:exp_avg_var_l2d_fred}
\begin{tabular}{@{}lc@{}}
\toprule
Method & Average Cumulative Cost \\
\midrule
\textbf{L2D-SLDS}
& \(\mathbf{0.004327 \pm 0.000003}\) \\
L2D-SLDS w/o \(\mathbf{g}_t\)
& \(0.004411 \pm 0.000011\) \\
\midrule
LinUCB
& \(0.004452 \pm 0.000002\) \\
NeuralUCB
& \(0.004424 \pm 0.000023\) \\
Random
& \(0.004455 \pm 0.000009\) \\
\midrule
Always expert 0
& \(0.004411\) \\
Always expert 1
& \(0.004567\) \\
Always expert 2
& \(0.004505\) \\
Always expert 3
& \(0.004329\) \\
Oracle
& \(0.001754\) \\
\bottomrule
\end{tabular}
\end{table}

\section{Conclusion}


\section{Impact Statement}

\bibliography{biblio}
\bibliographystyle{icml2026}

\appendix
\onecolumn

\section{Appendix Roadmap}
\label{app:roadmap}
This appendix collects (i) implementation-ready algorithms for the router and learning routines,
(ii) derivations underlying our exploration score, and (iii) proofs deferred from the main text.
It is organized as follows:
\begin{itemize}
        \item Appendix~\ref{app:notation}: notation table for the main paper.
	    \item Section~\ref{algo}: end-to-end router/filtering pseudocode and optional learning updates.
	    \item Appendix~\ref{app:cross_covariance}: exact (non-factorized) Kalman update cross-covariance for the queried update.
	    \item Section~\ref{app:info_gain}: information-gain derivations used for IDS-style exploration.
	    \item Appendix~\ref{app:proof_cross_update}--\ref{app:transfer}: proofs of propositions.
\end{itemize}

\section{Notation}
\label{app:notation}
\renewcommand{\arraystretch}{1.15}
\begin{longtable}{@{}p{0.22\linewidth}p{0.74\linewidth}@{}}
\toprule
\textbf{Symbol} & \textbf{Meaning} \\
\midrule
\endfirsthead
\toprule
\textbf{Symbol} & \textbf{Meaning} \\
\midrule
\endhead
\bottomrule
\endfoot
\bottomrule
\endlastfoot

\multicolumn{2}{@{}l@{}}{\textbf{Time, data, and actions}} \\
\midrule
\(t\in[T]\) & Round index; finite horizon \(T\). \\
\(\mathbf{x}_t\in\mathbb{R}^d\) & Observed context at round \(t\). \\
\(\vy_t\in\mathbb{R}^{d_y}\) & Target/label at round \(t\). \\
\(\mathcal{E}_t\) & Set of available experts at round \(t\) (may vary with time). \\
\(I_t\in\mathcal{E}_t\) & Queried expert at round \(t\). \\
\(\vhaty_{t,k}\in\mathbb{R}^{d_y}\) & Prediction of expert \(k\) at round \(t\). \\
\(O_t=(I_t,\vhaty_{t,I_t},\vy_t)\) & Post-action feedback tuple at round \(t\). \\
\(\mathcal{H}_t\) & Interaction history through the end of round \(t\). \\
\(\mathcal{F}_t\) & Decision-time sigma-algebra (information before choosing \(I_t\)). \\

\midrule
\multicolumn{2}{@{}l@{}}{\textbf{Residuals, costs, and objective}} \\
\midrule
\(e_{t,k}=\vhaty_{t,k}-\vy_t\) & Signed residual of expert \(k\) at time \(t\); realized residual \(e_t=e_{t,I_t}\). \\
\(\psi(\cdot)\) & Convex loss applied to residuals (e.g., \(\lVert\cdot\rVert_2^2\)). \\
\(\beta_k\ge 0\) & Expert-specific query fee. \\
\(C_{t,k}=\psi(e_{t,k})+\beta_k\) & Routing cost for expert \(k\); realized cost \(C_t=C_{t,I_t}\). \\
\(J(\pi)=\mathbb{E}\!\left[\sum_{t=1}^T C_{t,I_t}\right]\) & Expected cumulative cost of policy \(\pi\). \\
\(k_t^\star\) & Myopic Bayes benchmark minimizing \(\mathbb{E}[C_{t,k}\mid\mathcal{F}_t]\) over \(k\in\mathcal{E}_t\). \\

\midrule
\multicolumn{2}{@{}l@{}}{\textbf{Latent-state model (factorized switching LDS)}} \\
\midrule
\(z_t\in\{1,\dots,M\}\) & Discrete latent regime at round \(t\); \(M\) regimes. \\
\(\Pi_\theta(\mathbf{x}_t)\in[0,1]^{M\times M}\) & Context-dependent transition matrix; \(\mathbb{P}(z_t=m\mid z_{t-1}=\ell,\mathbf{x}_t)=\Pi_\theta(\mathbf{x}_t)_{\ell m}\). \\
\(\theta\) & Parameters of the context-dependent transition model \(\Pi_\theta(\mathbf{x}_t)\). \\
\(d_{\mathrm{attn}}\) & Bottleneck dimension in the low-rank transition-parameterization. \\
\(\mathbf{g}_t\in\mathbb{R}^{d_g}\) & Shared global latent factor coupling experts. \\
\(\mathbf{u}_{t,k}\in\mathbb{R}^{d_\alpha}\) & Expert-specific idiosyncratic latent state. \\
\(\mathbf{A}^{(g)}_m,\mathbf{Q}^{(g)}_m\) & Regime-\(m\) dynamics matrix and process noise covariance for \(\mathbf{g}_t\). \\
\(\mathbf{A}^{(u)}_m,\mathbf{Q}^{(u)}_m\) & Regime-\(m\) dynamics matrix and process noise covariance for \(\mathbf{u}_{t,k}\) (shared across experts). \\
\(\Phi(\mathbf{x}_t)\) & Feature map used in the residual emission mean. \\
\(\mathbf{B}_k\in\mathbb{R}^{d_\alpha\times d_g}\) & Expert-specific loading matrix coupling \(\mathbf{g}_t\) into expert \(k\)'s residual model. \\
\(\boldsymbol\alpha_{t,k}=\mathbf{B}_k\mathbf{g}_t+\mathbf{u}_{t,k}\) & Latent ``reliability'' vector of expert \(k\) at time \(t\). \\
\(\mathbf{R}_{m,k}\in\mathbb{S}^{d_y}_{++}\) & Regime- and expert-specific emission noise covariance. \\
\(\Theta\) & Collection of model parameters (e.g., \(\Pi_\theta\), \((\mathbf{A}^{(g)}_m,\mathbf{Q}^{(g)}_m)_m\), \((\mathbf{A}^{(u)}_m,\mathbf{Q}^{(u)}_m)_m\), \((\mathbf{B}_k)_k\), \((\mathbf{R}_{m,k})_{m,k}\)). \\

\midrule
\multicolumn{2}{@{}l@{}}{\textbf{Filtering, prediction, and routing scores}} \\
\midrule
\(\bar w_t^{(m)}=\mathbb{P}(z_t=m\mid\mathcal{F}_t)\) & Predictive (pre-observation) regime weight. \\
\(w_t^{(m)}=\mathbb{P}(z_t=m\mid\mathcal{F}_t,I_t,e_t)\) & Filtering (post-observation) regime weight. \\
\(\gamma_t^{(m)}\) & Posterior regime responsibility used in (Monte Carlo) EM. \\
\(\xi_{t-1}^{(\ell,m)}\) & Posterior transition responsibility used in (Monte Carlo) EM. \\
\(e_{t,k}^{\mathrm{pred}}\) & One-step-ahead predictive residual random variable for expert \(k\). \\
\(\bar C_{t,k}^{\mathrm{pred}}\) & Predicted cost: \(\mathbb{E}[\psi(e_{t,k}^{\mathrm{pred}})\mid\mathcal{F}_t]+\beta_k\). \\
\(k_t^{\mathrm{pred}}\) & Myopic predicted-cost minimizer in \(\mathcal{E}_t\). \\
\(\Delta_t(k)\) & Predicted cost gap relative to \(k_t^{\mathrm{pred}}\). \\
\(\mathrm{IG}_t(k)\) & Information gain: \(\mathcal{I}((z_t,\mathbf{g}_t);e_{t,k}^{\mathrm{pred}}\mid\mathcal{F}_t)\). \\
\(\epsilon_w\) & Mixing floor for predictive mode weights \(\bar w_t^{(m)}\) in IMM updates. \\
\(\epsilon_{\mathrm{IG}}\) & Information-gain floor used in IDS (avoids division by zero and clamps Monte Carlo noise). \\
\(S\) & Monte Carlo sample size used to estimate the mode-identification term in \(\mathrm{IG}_t(k)\). \\

\midrule
\multicolumn{2}{@{}l@{}}{\textbf{Dynamic registry management}} \\
\midrule
\(\mathcal{K}_t\) & Maintained expert registry: experts for which per-expert filtering marginals (hence \(\mathbf{u}_{t,k}\)) are stored. \\
\(\mathcal{E}_t^{\mathrm{init}}=\mathcal{E}_t\setminus \mathcal{K}_{t-1}\) & Entering experts at round \(t\) (new or re-entering after pruning). \\
\(\tau_{\mathrm{last}}(k)\) & Last round at which expert \(k\) was queried. \\
\(\Delta_{\max}\) & Staleness horizon controlling pruning. \\
\(\mathcal{K}_t^{\mathrm{stale}}\) & Stale experts eligible for pruning. \\
\end{longtable}
\renewcommand{\arraystretch}{1}

\section{L2D-SLDS Probabilistic Model}
We report the complete probabilistic graphical model of our L2D-SLDS with censored feedback and context-dependent
regime switching in Figure~\ref{fig:slds_pgm_final}.
\begin{figure}[H]
    \centering
%    \resizebox{0.25\textwidth}{!}{
    \begin{tikzpicture}[
    node distance=1.5cm and 3cm,
    >=latex,
    thick,
    latent_node/.style={latent, minimum size=1cm},
    obs_node/.style={obs, minimum size=1cm},
    rcurinfo_node/.style={latent, minimum size=1cm},
    action_node/.style={draw, rectangle, minimum size=0.9cm},
    param_edge/.style={->, dashed, color=gray!80}
]

% --- NODES ---
\node[latent_node] (zt_prev) {$z_{t-1}$};
\node[latent_node, right=of zt_prev] (zt) {$z_t$};
\node[latent_node, right=of zt] (zt_next) {$z_{t+1}$};

\node[obs_node, above=1.2cm of zt] (xt) {$\mathbf{x}_t$};

\node[latent_node, below=1.8cm of zt_prev] (gt_prev) {$\mathbf{g}_{t-1}$};
\node[latent_node, right=of gt_prev] (gt) {$\mathbf{g}_t$};

\node[latent_node, below=1.8cm of gt_prev] (ut_prev) {$\mathbf{u}_{t-1,j}$};
\node[latent_node, right=of ut_prev] (ut) {$\mathbf{u}_{t,j}$};

\node[obs_node, below=1.5cm of ut] (lt) {$e_{t,j}$};

% Plate over maintained registry
\plate {experts} {(ut_prev)(ut)(lt)} {$j \in \mathcal{K}_t$};

% Availability and action
\node[obs_node, right=4cm of ut] (Kt) {$\mc{E}_t$};
\node[action_node, below=1.5cm of Kt] (rt) {$I_t$};

% --- EDGES ---
\edge {zt_prev} {zt};
\edge {zt} {zt_next};
\edge {gt_prev} {gt};
\edge {ut_prev} {ut};

\draw[param_edge] (zt) to [bend right=20] node[pos=0.3, midway, right, font=\tiny] {$A^{(g)}_{z_t},Q^{(g)}_{z_t}$} (gt);
\draw[param_edge] (zt.south east) to [bend left=45] node[pos=0.7, right, font=\tiny, xshift=2pt] {$A^{(u)}_{z_t},Q^{(u)}_{z_t}$} (ut.north east);

\draw[->] (gt) to [bend left=45] (lt);
\edge {ut} {lt};

\draw[param_edge]
  (xt)
  to
  node[pos=0.5, right, font=\tiny, xshift=2pt]
  {$\Pi_\theta(\cdot,\cdot\mid \mathbf{x}_t)$}
  (zt);

\draw[->] (xt.east) to [out=0, in=0, looseness=1] node[midway, right, font=\small] {$\Phi(\mathbf{x}_t)$} (lt.east);

\edge {Kt} {rt};

% Selection affects what is observed (not what is generated)
\draw[->, dotted] (rt) -- (lt) node[midway, above, font=\tiny] {reveals};

\end{tikzpicture}
%}
    \caption{L2D-SLDS with bandit feedback and \emph{context-dependent} regime switching:
\(p(z_t\mid z_{t-1},\mathbf{x}_t)\). The plate \(j\in\mathcal{K}_t\) indexes experts whose
idiosyncratic states are stored. Each \(e_{t,j}\) is a \emph{potential} residual, but only \(e_{t,I_t}\)
is revealed at round \(t\).}
    \label{fig:slds_pgm_final}
\end{figure}

\section{Algorithms} \label{algo}
% =========================
% Router / Filtering Algorithms
% =========================
\subsection{Router and Filtering Recursion}
\label{app:alg_router}
\paragraph{Scope.} This subsection provides implementation-ready pseudocode for the per-round router
(Algorithm~\ref{alg:router_main}) and the queried update (Algorithm~\ref{alg:correct_reweight}). We
assume parameters \(\Theta\) and an initial belief are provided (learnable via Algorithm~\ref{alg:trainmodel_em}).

\begin{algorithm}[H]
\caption{Context-Aware Router (Factorized SLDS + IMM + IDS)}
\label{alg:router_main}
\begin{algorithmic}[1]
\STATE {\bfseries Input:} horizon \(T\); parameters \(\Theta\); feature map \(\Phi\); loss \(\psi\); fees \((\beta_k)_k\); default entry priors \((\mu^{(m)}_{\mathrm{init,def}},\Sigma^{(m)}_{\mathrm{init,def}})_{m=1}^M\); staleness \(\Delta_{\max}\); floors \((\epsilon_w,\epsilon_{\mathrm{IG}})\); Monte Carlo budget \(S\) for \(\mathrm{IG}_t(k)\) (Appendix~\ref{app:info_gain}).
\STATE {\bfseries Initialize:} \(w_0^{(m)}\leftarrow \mathbb{P}(z_1=m)\); \((\mu^{(m)}_{g,0|0},\Sigma^{(m)}_{g,0|0})_{m=1}^M\); \(\mathcal{K}_0\leftarrow\varnothing\); \(\tau_{\mathrm{last}}(k)\leftarrow 0\) for all \(k\).
\FOR{\(t=1\) to \(T\)}
    \STATE Observe \((\mathbf{x}_t,\mathcal{E}_t)\).
    \STATE \textbf{Registry:} \(\mathcal{E}^{\mathrm{init}}_t \leftarrow \mathcal{E}_t \setminus \mathcal{K}_{t-1}\).
    \STATE \textbf{Registry:} \(\mathcal{K}^{\mathrm{stale}}_t \leftarrow \{k\in\mathcal{K}_{t-1}\setminus\mathcal{E}_t:\ t-\tau_{\mathrm{last}}(k)>\Delta_{\max}\}\).
    \STATE \textbf{Registry:} \(\mathcal{K}_t \leftarrow (\mathcal{K}_{t-1}\cup \mathcal{E}_t)\setminus \mathcal{K}^{\mathrm{stale}}_t\). \COMMENT{Prune stale \(\mathbf{u}_{\cdot,k}\) marginals}
    \STATE For each \(k\in\mathcal{E}^{\mathrm{init}}_t\), set \((\mu^{(m)}_{\mathrm{init},k},\Sigma^{(m)}_{\mathrm{init},k})_{m=1}^M\) (default: \((\mu^{(m)}_{\mathrm{init,def}},\Sigma^{(m)}_{\mathrm{init,def}})\)).

    \STATE \COMMENT{\textbf{IMM predictive step:} compute \(\bar w_t^{(m)}=\mathbb{P}(z_t=m\mid\mathcal{F}_t)\) from \(w_{t-1}\) and \(\Pi_\theta(\mathbf{x}_t)\) (Eq.~\ref{eq:context_transitions}), with flooring \(\epsilon_w\), and moment-match mixed priors at time \(t-1\).}
    \STATE \COMMENT{\textbf{Time update:} apply Eqs.~\ref{eq:global_dynamics}, \ref{eq:idiosyncratic_dynamics}, and \ref{eq:birth_time_update} to obtain \((\mu^{(m)}_{g,t|t-1},\Sigma^{(m)}_{g,t|t-1})\) and \((\mu^{(m)}_{u,k,t|t-1},\Sigma^{(m)}_{u,k,t|t-1})\) for \(k\in\mathcal{K}_t\).}

    \STATE For each \(m\in[M]\) and \(k\in\mathcal{E}_t\), compute \((\bar e_{t,k}^{\mathrm{pred},(m)},\Sigma_{t,k}^{\mathrm{pred},(m)})\) from Eq.~\ref{eq:residual_emission}.
    \STATE For each \(k\in\mathcal{E}_t\), set \(\bar C_{t,k}^{\mathrm{pred}}\leftarrow \sum_{m=1}^M \bar w_t^{(m)}\big(\mathbb{E}_{e\sim\mathcal{N}(\bar e_{t,k}^{\mathrm{pred},(m)},\Sigma_{t,k}^{\mathrm{pred},(m)})}[\psi(e)]+\beta_k\big)\).
    \STATE \(k_t^{\mathrm{pred}} \in \arg\min_{k\in\mathcal{E}_t}\bar C_{t,k}^{\mathrm{pred}}\); \(\Delta_t(k)\leftarrow \bar C_{t,k}^{\mathrm{pred}}-\bar C_{t,k_t^{\mathrm{pred}}}^{\mathrm{pred}}\) for all \(k\in\mathcal{E}_t\).
    \STATE Compute \(\mathrm{IG}_t(k)=\mathcal{I}((z_t,\mathbf{g}_t);e_{t,k}^{\mathrm{pred}}\mid\mathcal{F}_t)\) as in Appendix~\ref{app:info_gain}; clamp \(\mathrm{IG}_t(k)\leftarrow \max(\mathrm{IG}_t(k),\epsilon_{\mathrm{IG}})\).
    \STATE Choose \(I_t \in \arg\min_{k\in\mathcal{E}_t}\Delta_t(k)^2/\mathrm{IG}_t(k)\).
    \STATE Observe \((\vhaty_{t,I_t},\vy_t)\), set \(e_t\leftarrow \vhaty_{t,I_t}-\vy_t\), and update \(\tau_{\mathrm{last}}(I_t)\leftarrow t\).
    \STATE Run Algorithm~\ref{alg:correct_reweight} to obtain \(w_t\) and updated posteriors for \(\mathbf{g}_t\) and \((\mathbf{u}_{t,k})_{k\in\mathcal{K}_t}\).
    \STATE Optional: update \(\Theta\) via Algorithm~\ref{alg:online_em}.
\ENDFOR
\end{algorithmic}
\end{algorithm}

\begin{algorithm}[H]
\caption{\textsc{Correct}: Queried Kalman Update and Mode Posterior}
\label{alg:correct_reweight}
\begin{algorithmic}[1]
\STATE {\bfseries Input:} \(\mathbf{x}_t\), queried residual \(e_t\), queried expert \(I_t\); predictive weights \(\bar w_t\); predictive states \(\{\mu^{(m)}_{g,t|t-1},\Sigma^{(m)}_{g,t|t-1}\}_{m=1}^M\) and \(\{\mu^{(m)}_{u,k,t|t-1},\Sigma^{(m)}_{u,k,t|t-1}\}_{m\in[M],\,k\in\mathcal{K}_t}\); parameters \((\mathbf{B}_{I_t},(\mathbf{R}_{m,I_t})_{m=1}^M)\).
\STATE \(H_t \leftarrow [\Phi(\mathbf{x}_t)^\top \mathbf{B}_{I_t}\;\;\Phi(\mathbf{x}_t)^\top]\).
\FOR{\(m=1\) to \(M\)}
    \STATE \(\mu^{(m)}_{s,t|t-1} \leftarrow [(\mu^{(m)}_{g,t|t-1})^\top\;(\mu^{(m)}_{u,I_t,t|t-1})^\top]^\top\).
    \STATE \(\Sigma^{(m)}_{s,t|t-1} \leftarrow \mathrm{diag}(\Sigma^{(m)}_{g,t|t-1},\Sigma^{(m)}_{u,I_t,t|t-1})\).
    \STATE \(\bar e_{t,I_t}^{\mathrm{pred},(m)} \leftarrow H_t\mu^{(m)}_{s,t|t-1}\),\quad
    \(\Sigma_{t,I_t}^{\mathrm{pred},(m)} \leftarrow H_t\Sigma^{(m)}_{s,t|t-1}H_t^\top + \mathbf{R}_{m,I_t}\).
    \STATE \(K_t^{(m)} \leftarrow \Sigma^{(m)}_{s,t|t-1}H_t^\top(\Sigma_{t,I_t}^{\mathrm{pred},(m)})^{-1}\).
    \STATE \(\mu^{(m)}_{s,t|t} \leftarrow \mu^{(m)}_{s,t|t-1} + K_t^{(m)}(e_t-\bar e_{t,I_t}^{\mathrm{pred},(m)})\).
    \STATE \(\Sigma^{(m)}_{s,t|t} \leftarrow \Sigma^{(m)}_{s,t|t-1} - K_t^{(m)}\Sigma_{t,I_t}^{\mathrm{pred},(m)}(K_t^{(m)})^\top\).
    \STATE Project to factorized marginals: keep only the diagonal blocks for \(\mathbf{g}_t\) and \(\mathbf{u}_{t,I_t}\); set \((\mu^{(m)}_{u,k,t|t},\Sigma^{(m)}_{u,k,t|t})\leftarrow(\mu^{(m)}_{u,k,t|t-1},\Sigma^{(m)}_{u,k,t|t-1})\) for \(k\neq I_t\).
    \STATE \(\mathcal{L}_t^{(m)} \leftarrow \mathcal{N}\!\big(e_t;\bar e_{t,I_t}^{\mathrm{pred},(m)},\Sigma_{t,I_t}^{\mathrm{pred},(m)}\big)\).
\ENDFOR
\STATE \(w_t^{(m)} \leftarrow \frac{\mathcal{L}_t^{(m)}\bar w_t^{(m)}}{\sum_{\ell=1}^M \mathcal{L}_t^{(\ell)}\bar w_t^{(\ell)}}\) for all \(m\in[M]\).
\STATE {\bfseries Return:} \(w_t\) and updated posteriors \(\{\mu^{(m)}_{g,t|t},\Sigma^{(m)}_{g,t|t}\}_{m=1}^M\), \(\{\mu^{(m)}_{u,k,t|t},\Sigma^{(m)}_{u,k,t|t}\}_{m\in[M],\,k\in\mathcal{K}_t}\).
\end{algorithmic}
\end{algorithm}

% =========================
% Parameter Learning / Adaptation
% =========================
\subsection{Parameter Learning and Online Adaptation}
\label{app:alg_learning}
\paragraph{Scope.} This subsection describes optional model-learning routines (offline initialization and
sliding-window adaptation). The main router only requires a filtering belief and the learned parameters.

\begin{algorithm}[H]
\caption{\textsc{LearnParameters\_MCEM}: Monte Carlo EM for the Factorized SLDS (windowed batch)}
\label{alg:trainmodel_em}
\begin{algorithmic}[1]
\STATE {\bfseries Input:} window $\mathcal{T}=\{t_a,\dots,t_b\}$; stream $(\mathbf{x}_t,I_t,e_t)_{t\in\mathcal{T}}$ with $e_t=\vhaty_{t,I_t}-\vy_t$; feature map $\Phi$; EM iterations $N_{\mathrm{EM}}$; MCMC settings $(N_{\mathrm{samp}},N_{\mathrm{burn}})$; occupancy floor $\epsilon_N>0$; (optional) regularization $(\lambda_\theta,\lambda_B)$ for $(\Pi_\theta,\mathbf{B})$.
\STATE $\mathcal{K}^{\mathrm{qry}}_{\mathcal{T}}\leftarrow \{I_t:\ t\in\mathcal{T}\}$. \COMMENT{Experts queried in the window}
\STATE {\bfseries Initialize:} parameters $\Theta^{(0)}$ and priors for $z_{t_a}$, $\mathbf{g}_{t_a}$, and $\{\mathbf{u}_{t_a,k}\}_{k\in\mathcal{K}^{\mathrm{qry}}_{\mathcal{T}}}$.
\FOR{iteration $i = 1$ to $N_{\mathrm{EM}}$}
	    \STATE \textbf{// E-step: Monte Carlo posterior (blocked Gibbs)}
	    \STATE Draw samples from \(p(z_{t_a:t_b},\mathbf{g}_{t_a:t_b},(\mathbf{u}_{t_a:t_b,k})_{k\in\mathcal{K}^{\mathrm{qry}}_{\mathcal{T}}}\mid (\mathbf{x}_t,I_t,e_t)_{t\in\mathcal{T}},\Theta^{(i-1)})\) by alternating:
	    \STATE \hspace{1.5em}1) sample \(z_{t_a:t_b}\) via FFBS from the conditional HMM given \(\mathbf{g}_{t_a:t_b}\) and \((\mathbf{u}_{t_a:t_b,k})_k\);
	    \STATE \hspace{1.5em}2) sample \(\mathbf{g}_{t_a:t_b}\) via Kalman smoothing given \(z_{t_a:t_b}\) and \((\mathbf{u}_{t,I_t})_{t\in\mathcal{T}}\);
	    \STATE \hspace{1.5em}3) for each \(k\in\mathcal{K}^{\mathrm{qry}}_{\mathcal{T}}\), sample \(\mathbf{u}_{t_a:t_b,k}\) via Kalman smoothing using only \(\{(t,e_t): I_t=k\}\).
	    \STATE From post-burn-in samples, estimate \(\gamma_t^{(m)}\approx \mathbb{P}(z_t=m\mid\cdot)\), \(\xi_{t-1}^{(\ell,m)}\approx \mathbb{P}(z_{t-1}=\ell,z_t=m\mid\cdot)\), and the moments used in the M-step.

	    \STATE \textbf{// M-step: MAP / regularized updates (factorized moments)}
	    \STATE Update \((\mathbf{A}^{(g)}_{m},\mathbf{Q}^{(g)}_{m})_{m=1}^M\) and \((\mathbf{A}^{(u)}_{m},\mathbf{Q}^{(u)}_{m})_{m=1}^M\) using weighted least-squares/covariance matching (skip updates when the effective count is \(\le\epsilon_N\); see below).
	    \STATE Update \((\mathbf{B}_k)_{k\in\mathcal{K}^{\mathrm{qry}}_{\mathcal{T}}}\) and \((\mathbf{R}_{m,k})_{m\in[M],\,k\in\mathcal{K}^{\mathrm{qry}}_{\mathcal{T}}}\) via weighted linear-Gaussian regression (skip updates when the effective count is \(\le\epsilon_N\); see below).
	    \STATE Update \(\theta\) by maximizing \(\sum_{t\in\mathcal{T}\setminus\{t_a\}}\sum_{\ell,m}\xi_{t-1}^{(\ell,m)}\log \Pi_\theta(\mathbf{x}_t)_{\ell m}-\frac{\lambda_\theta}{2}\lVert\theta\rVert_2^2\).
\ENDFOR
\STATE {\bfseries Return:} $\Theta^{(N_\mathrm{EM})}$
\end{algorithmic}
\end{algorithm}

\paragraph{Implementation notes (E-step).}
In step 1, FFBS samples \(z_{t_a:t_b}\) from the conditional distribution induced by the Markov
transition \(\Pi_\theta(\mathbf{x}_t)\) (Eq.~\ref{eq:context_transitions}) and the linear-Gaussian
dynamics/emission terms (Eqs.~\ref{eq:global_dynamics}, \ref{eq:idiosyncratic_dynamics},
\ref{eq:residual_emission}) evaluated at the current \(\mathbf{g}_{t_a:t_b}\) and
\((\mathbf{u}_{t_a:t_b,k})_k\). In step 2, conditioned on \(z_{t_a:t_b}\) and \((\mathbf{u}_{t,I_t})_t\),
the observation model for \(\mathbf{g}_t\) is
\(e_t-\Phi(\mathbf{x}_t)^\top\mathbf{u}_{t,I_t}=\Phi(\mathbf{x}_t)^\top\mathbf{B}_{I_t}\mathbf{g}_t+v_t\)
with \(v_t\sim\mathcal{N}(\mathbf{0},\mathbf{R}_{z_t,I_t})\). In step 3, for a fixed expert \(k\),
conditioning on \(z_{t_a:t_b}\) and \(\mathbf{g}_{t_a:t_b}\), the observations at times \(\{t:I_t=k\}\)
satisfy \(e_t-\Phi(\mathbf{x}_t)^\top\mathbf{B}_{k}\mathbf{g}_t=\Phi(\mathbf{x}_t)^\top\mathbf{u}_{t,k}+v_t\)
with the same \(v_t\).

\paragraph{M-step updates.}
Let \(\langle\cdot\rangle\) denote the average over post-burn-in samples.
For each regime \(m\), define \(N_m\coloneqq\sum_{t=t_a+1}^{t_b}\gamma_t^{(m)}\) and the sufficient
statistics
\[
S^{(m)}_{g g^-}\coloneqq \sum_{t=t_a+1}^{t_b}\left\langle \mathbf{1}\{z_t=m\}\mathbf{g}_t\mathbf{g}_{t-1}^\top\right\rangle,
\qquad
S^{(m)}_{g^- g^-}\coloneqq \sum_{t=t_a+1}^{t_b}\left\langle \mathbf{1}\{z_t=m\}\mathbf{g}_{t-1}\mathbf{g}_{t-1}^\top\right\rangle.
\]
If \(N_m>\epsilon_N\), set \(\mathbf{A}^{(g)}_{m}\leftarrow S^{(m)}_{g g^-}\left(S^{(m)}_{g^- g^-}\right)^{-1}\) and
\[
\mathbf{Q}^{(g)}_{m}
\leftarrow
\frac{1}{N_m}\sum_{t=t_a+1}^{t_b}\left\langle \mathbf{1}\{z_t=m\}\left(\mathbf{g}_t-\mathbf{A}^{(g)}_m\mathbf{g}_{t-1}\right)\left(\mathbf{g}_t-\mathbf{A}^{(g)}_m\mathbf{g}_{t-1}\right)^\top\right\rangle.
\]
Define \(N_m^{(u)}\coloneqq\sum_{t=t_a+1}^{t_b}\sum_{k\in\mathcal{K}^{\mathrm{qry}}_{\mathcal{T}}}\gamma_t^{(m)}\) and
\[
S^{(m)}_{u u^-}\coloneqq \sum_{t=t_a+1}^{t_b}\sum_{k\in\mathcal{K}^{\mathrm{qry}}_{\mathcal{T}}}\left\langle \mathbf{1}\{z_t=m\}\mathbf{u}_{t,k}\mathbf{u}_{t-1,k}^\top\right\rangle,
\qquad
S^{(m)}_{u^- u^-}\coloneqq \sum_{t=t_a+1}^{t_b}\sum_{k\in\mathcal{K}^{\mathrm{qry}}_{\mathcal{T}}}\left\langle \mathbf{1}\{z_t=m\}\mathbf{u}_{t-1,k}\mathbf{u}_{t-1,k}^\top\right\rangle.
\]
If \(N_m^{(u)}>\epsilon_N\), set \(\mathbf{A}^{(u)}_{m}\leftarrow S^{(m)}_{u u^-}\left(S^{(m)}_{u^- u^-}\right)^{-1}\) and
\[
\mathbf{Q}^{(u)}_{m}
\leftarrow
\frac{1}{N_m^{(u)}}\sum_{t=t_a+1}^{t_b}\sum_{k\in\mathcal{K}^{\mathrm{qry}}_{\mathcal{T}}}\left\langle \mathbf{1}\{z_t=m\}\left(\mathbf{u}_{t,k}-\mathbf{A}^{(u)}_m\mathbf{u}_{t-1,k}\right)\left(\mathbf{u}_{t,k}-\mathbf{A}^{(u)}_m\mathbf{u}_{t-1,k}\right)^\top\right\rangle.
\]

\textbf{Emission parameters \((\mathbf{B}_k,\mathbf{R}_{m,k})\).}
Fix an expert \(k\in\mathcal{K}^{\mathrm{qry}}_{\mathcal{T}}\) and denote \(\Phi_t\coloneqq \Phi(\mathbf{x}_t)\).
For each \(t\in\mathcal{T}\) with \(I_t=k\), define the residual after removing the idiosyncratic term
\(y_t\coloneqq e_t-\Phi_t^\top \mathbf{u}_{t,k}\in\mathbb{R}^{d_y}\) and the design matrix
\(X_t\coloneqq (\mathbf{g}_t^\top\otimes \Phi_t^\top)\in\mathbb{R}^{d_y\times (d_g d_\alpha)}\),
so that \(y_t=X_t\,\mathrm{vec}(\mathbf{B}_k)+v_t\) with \(v_t\sim\mathcal{N}(\mathbf{0},\mathbf{R}_{z_t,k})\).
Here \(\otimes\) is the Kronecker product and \(\mathrm{vec}(\cdot)\) stacks matrix columns.
Given current \((\mathbf{R}_{m,k})_{m=1}^M\), a (ridge) generalized least-squares update is
\[
\mathrm{vec}(\mathbf{B}_k)\leftarrow
\Big(\sum_{t\in\mathcal{T}:I_t=k}\sum_{m=1}^M \left\langle \mathbf{1}\{z_t=m\}\,X_t^\top \mathbf{R}_{m,k}^{-1} X_t\right\rangle+\lambda_B \mathbf{I}\Big)^{-1}
\Big(\sum_{t\in\mathcal{T}:I_t=k}\sum_{m=1}^M \left\langle \mathbf{1}\{z_t=m\}\,X_t^\top \mathbf{R}_{m,k}^{-1} y_t\right\rangle\Big).
\]
For each regime \(m\), define the effective count \(N_{m,k}\coloneqq\sum_{t\in\mathcal{T}:I_t=k}\gamma_t^{(m)}\).
If \(N_{m,k}>\epsilon_N\), update the emission covariance by weighted covariance matching:
\[
\mathbf{R}_{m,k}
\leftarrow
\frac{1}{N_{m,k}}\sum_{t\in\mathcal{T}:I_t=k}\left\langle \mathbf{1}\{z_t=m\}\,r_{t,k}r_{t,k}^\top\right\rangle,
\quad
r_{t,k}\coloneqq e_t-\Phi_t^\top(\mathbf{B}_k\mathbf{g}_t+\mathbf{u}_{t,k}).
\]

\begin{algorithm}[H]
\caption{\textsc{OnlineUpdate}: Sliding-Window Monte Carlo EM (non-stationary adaptation)}
\label{alg:online_em}
\begin{algorithmic}[1]
\STATE {\bfseries Input:} current time $t$; stream $(\mathbf{x}_\tau,\mathcal{E}_\tau,I_\tau,e_\tau)_{\tau\le t}$; current parameters $\Theta^{(t-1)}$; window length $W$; update period $K$; EM iterations $N_{\mathrm{EM}}^{\mathrm{win}}$; MCMC settings; occupancy floor $\epsilon_N$; hyperparameters as in Algorithm~\ref{alg:trainmodel_em}.
\STATE $\tau_0 \leftarrow t-W+1$.
\IF{$t < W$ \textbf{or} $t \bmod K \neq 0$}
    \STATE $\Theta^{(t)} \leftarrow \Theta^{(t-1)}$ and \textbf{return}.
\ENDIF
\STATE Define window $\mathcal{T}_t \leftarrow \{\tau_0,\dots,t\}$ and $\mathcal{K}^{\mathrm{qry}}_{\mathcal{T}_t}\leftarrow \{I_\tau:\ \tau\in\mathcal{T}_t\}$.
\STATE Initialize priors for $z_{\tau_0}$, $\mathbf{g}_{\tau_0}$, and $\{\mathbf{u}_{\tau_0,k}\}_{k\in\mathcal{K}^{\mathrm{qry}}_{\mathcal{T}_t}}$ from the stored filtering belief at time $\tau_0-1$ (plus one time-update); if unavailable, use conservative default priors.
\STATE Run Algorithm~\ref{alg:trainmodel_em} on $\mathcal{T}_t$ with initialization $\Theta^{(t-1)}$, floor $\epsilon_N$, and $N_{\mathrm{EM}}^{\mathrm{win}}$ iterations.
\STATE Re-run a forward filtering pass over $\mathcal{T}_t$ under $\Theta^{(t)}$ to refresh the belief at time $t$ (starting from the window-initial prior).
\STATE {\bfseries Return:} updated parameters $\Theta^{(t)}$.
\end{algorithmic}
\end{algorithm}

% =========================
% Additional Derivations
% =========================
\subsection{Cross-Covariance in the Exact Update}
\label{app:cross_covariance}

The Kalman update in Algorithm~\ref{alg:correct_reweight} is performed on the joint state
\(\mathbf{s}_t \coloneqq (\mathbf{g}_t,\mathbf{u}_{t,I_t})\). For readability in this subsection, set
\(\mathbf{u}_t\coloneqq \mathbf{u}_{t,I_t}\) and write \(\mathbf{s}_t=(\mathbf{g}_t,\mathbf{u}_t)\).
Even if the predictive covariance is
block-diagonal (our factorized predictive belief), the \emph{exact} posterior covariance after
conditioning on the queried residual \(e_t\) generally has non-zero off-diagonal blocks:
\[
\Sigma^{(m)}_{s,t\mid t}
=
\begin{bmatrix}
\Sigma^{(m)}_{g,t\mid t} & \Sigma^{(m)}_{g u,t\mid t} \\
(\Sigma^{(m)}_{g u,t\mid t})^\top & \Sigma^{(m)}_{u,t\mid t}
\end{bmatrix},
\qquad
\Sigma^{(m)}_{g u,t\mid t}\neq \mathbf{0}\ \text{in general}.
\]
These cross terms arise because the observation matrix
\(H_t=[\Phi(\mathbf{x}_t)^\top\mathbf{B}_{I_t}\;\;\Phi(\mathbf{x}_t)^\top]\)
couples \(\mathbf{g}_t\) and \(\mathbf{u}_{t,I_t}\). Retaining \(\Sigma^{(m)}_{g u,t\mid t}\) would
propagate correlation into subsequent steps and into cross-expert predictive covariances.

\paragraph{Closed-form cross-covariance.}
Write the Kalman gain in block form
\(
K_t^{(m)}=\big[(K^{(m)}_{g,t})^\top\ (K^{(m)}_{u,t})^\top\big]^\top
\),
and let \(\Sigma_{t,I_t}^{\mathrm{pred},(m)}\) denote the innovation covariance of the queried residual
as in Algorithm~\ref{alg:correct_reweight}:
\(
\Sigma_{t,I_t}^{\mathrm{pred},(m)} = H_t\Sigma^{(m)}_{s,t\mid t-1}H_t^\top + \mathbf{R}_{m,I_t}.
\)
Then the covariance update can be written as
\(
\Sigma^{(m)}_{s,t\mid t}
=
\Sigma^{(m)}_{s,t\mid t-1}
-K_t^{(m)} \Sigma_{t,I_t}^{\mathrm{pred},(m)} (K_t^{(m)})^\top
\).
If the predictive covariance is block-diagonal, then the off-diagonal block is
\[
\Sigma^{(m)}_{g u,t\mid t}
=
-K^{(m)}_{g,t} \Sigma_{t,I_t}^{\mathrm{pred},(m)} (K^{(m)}_{u,t})^\top
=
-\Sigma^{(m)}_{g,t\mid t-1} H_{g,t}^\top (\Sigma_{t,I_t}^{\mathrm{pred},(m)})^{-1} H_{u,t}\Sigma^{(m)}_{u,t\mid t-1},
\]
where \(H_{g,t}=\Phi(\mathbf{x}_t)^\top\mathbf{B}_{I_t}\in\mathbb{R}^{d_y\times d_g}\) and
\(H_{u,t}=\Phi(\mathbf{x}_t)^\top\in\mathbb{R}^{d_y\times d_\alpha}\).
Unless one of these terms is zero, the cross-covariance is non-zero.

\paragraph{Why we discard it.}
Keeping \(\Sigma^{(m)}_{g u,t\mid t}\) is exact but undermines the factorized SLDS approximation that
enables scalable inference under a growing expert registry. Once \(\mathbf{g}_t\) becomes correlated
with \(\mathbf{u}_{t,I_t}\), future prediction steps introduce non-zero cross-covariances between
\(\mathbf{g}_t\) and every \(\mathbf{u}_{t,k}\) that shares dynamics with \(\mathbf{u}_{t,I_t}\), and,
through the shared factor, induce dependence across many experts. This breaks the stored-marginal
structure, increases both compute and memory (scaling with the full registry size), and complicates
closed-form quantities used in Section~\ref{sec:exploration} (e.g., the Gaussian channel form and
information gain). For these reasons, we project back to a factorized belief after each update and
retain only the diagonal blocks as in Algorithm~\ref{alg:correct_reweight}.

% =========================
% Information Gain Details
% =========================
\section{Information Gain for Exploration}
\label{app:info_gain}

\begin{remark}[$(z_t,\mathbf{g}_t)$-Information Gain for Non-Stationary Routing]
\label{rmk:zg_information}
Algorithm~\ref{alg:router_main} uses the \emph{full} $(z_t,\mathbf{g}_t)$-information gain rather than
conditioning only on $\mathbf{g}_t$. By the chain rule for mutual information:
\begin{equation}
\label{eq:ig_chain_rule_rmk}
\mathcal{I}\!\left((z_t,\mathbf{g}_t);e_{t,k}^{\mathrm{pred}}\,\middle|\,\mathcal{F}_t\right)
=
\underbrace{\mathcal{I}\!\left(z_t;e_{t,k}^{\mathrm{pred}}\,\middle|\,\mathcal{F}_t\right)}_{\text{mode-identification}}
+
\underbrace{\mathcal{I}\!\left(\mathbf{g}_t;e_{t,k}^{\mathrm{pred}}\,\middle|\,z_t,\mathcal{F}_t\right)}_{\text{shared-factor refinement}}.
\end{equation}
The first term measures how much observing the residual $e_{t,k}^{\mathrm{pred}}$ helps identify the current
regime $z_t$. This is crucial for non-stationarity: when modes have distinct predictive distributions, querying
an expert whose residual discriminates between regimes accelerates adaptation to regime changes.
\end{remark}
\textbf{Why both terms matter:}
\begin{itemize}
\item \emph{Shared-factor refinement} (closed-form): Reduces posterior uncertainty about $\mathbf{g}_t$,
improving predictions for \emph{all} experts via Proposition~\ref{prop:cross_update}.
\item \emph{Mode-identification} (Monte Carlo): Reduces uncertainty about $z_t$, ensuring the router uses
the correct dynamics parameters $(\mathbf{A}_m^{(g)},\mathbf{Q}_m^{(g)},\mathbf{A}_m^{(u)},\mathbf{Q}_m^{(u)},\mathbf{R}_{m,k})$.
\end{itemize}

\textbf{Computational note:} The mode-identification term requires Monte Carlo estimation because the
predictive distribution $p(e_{t,k}^{\mathrm{pred}}\mid\mathcal{F}_t)$ is a Gaussian mixture, for which KL divergence has no
closed form. The LogSumExp trick ensures numerical stability. With $S=50$ samples per expert, the overhead is
negligible compared to the SLDS update cost.


\subsection{Exploration via \((z_t,\mathbf{g}_t)\)-information}
\label{sec:exploration_zg}

Bandit feedback reveals only the queried expert's residual, so the router must trade off
\emph{exploitation} (low immediate cost) against \emph{learning} (reducing posterior uncertainty to
improve future decisions).
In our IMM-factorized SLDS, two latent objects drive both non-stationarity and cross-expert transfer:
the regime \(z_t\in\{1,\dots,M\}\) and the shared factor \(\mathbf{g}_t\) (Proposition~\ref{prop:cross_update}).
We therefore score exploration by the information gained about the \emph{joint} latent state
\((z_t,\mathbf{g}_t)\) from the (potential) queried residual.
Throughout, logarithms are natural unless stated otherwise, so mutual information is measured in nats
(replace \(\log\) by \(\log_2\) to obtain bits).
We reuse the core SLDS/IMM notation from the main text: \(\Phi(\mathbf{x}_t)\), \(\mathbf{B}_k\),
\(\bar{w}_t^{(m)}=\mathbb{P}(z_t=m\mid\mathcal{F}_t)\), and the predictive moments
\((\mu^{(m)}_{g,t\mid t-1},\Sigma^{(m)}_{g,t\mid t-1})\), \((\mu^{(m)}_{u,k,t\mid t-1},\Sigma^{(m)}_{u,k,t\mid t-1})\), and \(\mathbf{R}_{m,k}\).
For Monte Carlo, we use \(\tilde{\cdot}\) to denote sampled quantities and write
\(\tilde z\sim \mathrm{Cat}((\bar{w}_t^{(m)})_{m=1}^M)\) for a categorical draw from the mode weights.

\paragraph{Decision-time predictive random variables.}
At round \(t\), the decision-time sigma-algebra is
\(
\mathcal{F}_t=\sigma(\mathcal{H}_{t-1},\mathbf{x}_t,\mathcal{E}_t)
\)
and the router chooses \(I_t\in\mathcal{E}_t\).
For each available expert \(k\in\mathcal{E}_t\), define the pre-query predictive residual random variable
\begin{equation}
\label{eq:exp_pred_rv}
e_{t,k}^{\mathrm{pred}} \sim p(e_{t,k}\mid \mathcal{F}_t).
\end{equation}
If \(I_t=k\), the realized observation is \(e_t=e_{t,k}\) and
\(
e_t \mid (\mathcal{F}_t,I_t=k)\overset{d}{=} e_{t,k}^{\mathrm{pred}}\mid \mathcal{F}_t.
\)

\paragraph{Per-mode linear-Gaussian predictive parametrization (IMM outputs).}
Fix a regime \(z_t=m\).
The IMM predictive step yields a Gaussian predictive prior for the shared factor:
\begin{equation}
\label{eq:exp_g_prior}
\mathbf{g}_t\mid(\mathcal{F}_t,z_t=m)\sim
\mathcal{N}\!\left(\mu^{(m)}_{g,t\mid t-1},\,\Sigma^{(m)}_{g,t\mid t-1}\right).
\end{equation}
Under the factorized predictive belief, querying expert \(k\) induces the linear-Gaussian observation channel
\begin{equation}
\label{eq:exp_channel}
e_{t,k}^{\mathrm{pred}} \mid (\mathbf{g}_t,\mathcal{F}_t,z_t=m)
\sim
\mathcal{N}\!\big(\mathbf{H}_{t,k}\mathbf{g}_t + \mathbf{b}^{(m)}_{t,k},\, \mathbf{S}^{(m)}_{t,k}\big),
\end{equation}
with mode-specific quantities
\begin{equation}
\label{eq:exp_channel_params}
\begin{aligned}
\mathbf{H}_{t,k} &\coloneqq \Phi(\mathbf{x}_t)^\top \mathbf{B}_k\in\mathbb{R}^{d_y\times d_g},\\
\mathbf{b}^{(m)}_{t,k} &\coloneqq \Phi(\mathbf{x}_t)^\top \mu^{(m)}_{u,k,t\mid t-1}\in\mathbb{R}^{d_y},\\
\mathbf{S}^{(m)}_{t,k} &\coloneqq
\Phi(\mathbf{x}_t)^\top \Sigma^{(m)}_{u,k,t\mid t-1}\Phi(\mathbf{x}_t) + \mathbf{R}_{m,k}\in\mathbb{S}^{d_y}_{++}.
\end{aligned}
\end{equation}

\paragraph{Exploitation score: predictive cost and gap.}
Recall the realized cost \(C_{t,k}=\psi(e_{t,k})+\beta_k\), where \(\beta_k\ge 0\) is the known query
fee.
In practice, we use squared loss,
\begin{equation}
\label{eq:exp_squared_loss}
\psi(u)=\lVert u\rVert_2^2,
\end{equation}
and we will simplify expressions accordingly; nothing in the \((z_t,\mathbf{g}_t)\)-information score
depends on this choice.
Define the predictive (virtual) cost random variable
\begin{equation}
\label{eq:exp_virtual_cost}
C_{t,k}^{\mathrm{pred}}
\coloneqq
\psi(e_{t,k}^{\mathrm{pred}})+\beta_k,
\qquad k\in\mathcal{E}_t,
\end{equation}
with conditional mean
\begin{equation}
\label{eq:exp_mean_cost}
\bar C_{t,k}^{\mathrm{pred}}\coloneqq \mathbb{E}\!\left[C_{t,k}^{\mathrm{pred}}\,\middle|\,\mathcal{F}_t\right]
=
\mathbb{E}\!\left[\psi(e_{t,k}^{\mathrm{pred}})\,\middle|\,\mathcal{F}_t\right]+\beta_k.
\end{equation}
Let \(k_t^{\mathrm{pred}}\in\arg\min_{k\in\mathcal{E}_t}\bar C_{t,k}^{\mathrm{pred}}\) and define the predictive gap
\begin{equation}
\label{eq:exp_gap}
\Delta_t(k)\coloneqq \bar C_{t,k}^{\mathrm{pred}}-\bar C_{t,k_t^{\mathrm{pred}}}^{\mathrm{pred}}\ge 0.
\end{equation}

\paragraph{Computing \(\bar C_{t,k}^{\mathrm{pred}}\) from per-mode moments.}
From \eqref{eq:exp_g_prior}--\eqref{eq:exp_channel}, the mode-conditioned predictive residual is Gaussian with
\begin{align}
\label{eq:exp_residual_mean}
\bar e_{t,k}^{\mathrm{pred},(m)}
&\coloneqq
\mathbb{E}\!\left[e_{t,k}^{\mathrm{pred}}\,\middle|\,\mathcal{F}_t,z_t=m\right]
=
\mathbf{H}_{t,k}\mu^{(m)}_{g,t\mid t-1}+\mathbf{b}^{(m)}_{t,k}\in\mathbb{R}^{d_y},\\
\label{eq:exp_residual_var}
\Sigma_{t,k}^{\mathrm{pred},(m)}
&\coloneqq
\mathrm{Cov}\!\left(e_{t,k}^{\mathrm{pred}}\,\middle|\,\mathcal{F}_t,z_t=m\right)
=
\mathbf{H}_{t,k}\Sigma^{(m)}_{g,t\mid t-1}\mathbf{H}_{t,k}^\top+\mathbf{S}^{(m)}_{t,k}\in\mathbb{S}^{d_y}_{++}.
\end{align}
Let \(\bar{w}_t^{(m)}=\mathbb{P}(z_t=m\mid\mathcal{F}_t)\).
Then \(p(e_{t,k}^{\mathrm{pred}}\mid\mathcal{F}_t)=\sum_{m=1}^M \bar{w}_t^{(m)}\,\mathcal{N}(\bar e_{t,k}^{\mathrm{pred},(m)},\Sigma_{t,k}^{\mathrm{pred},(m)})\).
For general \(\psi\),
\begin{equation}
\label{eq:exp_cost_mix}
\mathbb{E}\!\left[\psi(e_{t,k}^{\mathrm{pred}})\,\middle|\,\mathcal{F}_t\right]
=
\sum_{m=1}^M \bar{w}_t^{(m)}\,
\mathbb{E}\!\left[\psi(E)\right]_{E\sim\mathcal{N}(\bar e_{t,k}^{\mathrm{pred},(m)},\Sigma_{t,k}^{\mathrm{pred},(m)})}.
\end{equation}
In the squared-loss case \(\psi(e)=\lVert e\rVert_2^2\) from \eqref{eq:exp_squared_loss}, we have
\(\mathbb{E}[\lVert E\rVert_2^2]=\mathrm{tr}(\Sigma)+\lVert \mu\rVert_2^2\), hence
\begin{equation}
\label{eq:exp_square_loss_mix}
\bar C_{t,k}^{\mathrm{pred}}
=
\left(\sum_{m=1}^M \bar{w}_t^{(m)}\big(\mathrm{tr}(\Sigma_{t,k}^{\mathrm{pred},(m)})+\lVert \bar e_{t,k}^{\mathrm{pred},(m)}\rVert_2^2\big)\right)+\beta_k.
\end{equation}

\paragraph{Learning score: information about \((z_t,\mathbf{g}_t)\).}
Define the \((z_t,\mathbf{g}_t)\)-information gain of querying expert \(k\) by
\begin{equation}
\label{eq:exp_ig_zg_def}
\mathrm{IG}_t(k)
\coloneqq
\mathcal{I}\!\left((z_t,\mathbf{g}_t);\ e_{t,k}^{\mathrm{pred}}\,\middle|\,\mathcal{F}_t\right).
\end{equation}
By the chain rule,
\begin{align}
\label{eq:exp_ig_zg_chain}
\mathrm{IG}_t(k)
&=
\mathcal{I}\!\left(z_t;\ e_{t,k}^{\mathrm{pred}}\,\middle|\,\mathcal{F}_t\right)
+
\mathcal{I}\!\left(\mathbf{g}_t;\ e_{t,k}^{\mathrm{pred}}\,\middle|\,\mathcal{F}_t,z_t\right)\\
&=
\underbrace{\mathcal{I}\!\left(z_t;\ e_{t,k}^{\mathrm{pred}}\,\middle|\,\mathcal{F}_t\right)}_{\text{mode-identification}}
+
\underbrace{\sum_{m=1}^M \bar{w}_t^{(m)}\,
\mathcal{I}\!\left(\mathbf{g}_t;\ e_{t,k}^{\mathrm{pred}}\,\middle|\,\mathcal{F}_t,z_t=m\right)}_{\text{shared-factor refinement}}.
\end{align}
The second term admits a closed form per mode; the first term is an information quantity for a
\(d_y\)-dimensional Gaussian mixture that can be computed accurately with light Monte Carlo.

\paragraph{Closed form: \(\mathcal{I}(\mathbf{g}_t;e_{t,k}^{\mathrm{pred}}\mid \mathcal{F}_t,z_t=m)\).}
Fix \(z_t=m\).
Let \(G\coloneqq \mathbf{g}_t\) and \(Y\coloneqq e_{t,k}^{\mathrm{pred}}\).
Equation \eqref{eq:exp_channel} implies the affine Gaussian channel
\(
Y=\mathbf{H}_{t,k}G+\mathbf{b}^{(m)}_{t,k}+\varepsilon
\)
with \(\varepsilon\sim\mathcal{N}(\mathbf{0},\mathbf{S}^{(m)}_{t,k})\) independent of \(G\).
Then
\begin{equation}
\label{eq:exp_ig_g_mode}
\mathcal{I}\!\left(\mathbf{g}_t;\ e_{t,k}^{\mathrm{pred}}\,\middle|\,\mathcal{F}_t,z_t=m\right)
=
\frac12\log\det\!\left(\mathbf{I}_{d_y}+\mathbf{H}_{t,k}\Sigma^{(m)}_{g,t\mid t-1}\mathbf{H}_{t,k}^\top(\mathbf{S}^{(m)}_{t,k})^{-1}\right).
\end{equation}

\paragraph{Monte Carlo: \(\mathcal{I}(z_t;e_{t,k}^{\mathrm{pred}}\mid\mathcal{F}_t)\) for a Gaussian mixture.}
Let \(p_m(e)\coloneqq p(e_{t,k}^{\mathrm{pred}}=e\mid \mathcal{F}_t,z_t=m)=\mathcal{N}(e;\bar e_{t,k}^{\mathrm{pred},(m)},\Sigma_{t,k}^{\mathrm{pred},(m)})\) and
\(
p_{\mathrm{mix}}(e)\coloneqq \sum_{m=1}^M \bar{w}_t^{(m)}p_m(e).
\)
Then
\begin{align}
\label{eq:exp_iz_mc}
\mathcal{I}\!\left(z_t;\ e_{t,k}^{\mathrm{pred}}\,\middle|\,\mathcal{F}_t\right)
&=
\sum_{m=1}^M \bar{w}_t^{(m)}\,\mathrm{KL}\!\left(p_m\ \middle\|\ p_{\mathrm{mix}}\right)\\
&=
\sum_{m=1}^M \bar{w}_t^{(m)}\,
\mathbb{E}_{E\sim p_m}\!\left[\log p_m(E)-\log p_{\mathrm{mix}}(E)\right].
\end{align}
This suggests the estimator (with \(S\) samples per mode):
\begin{equation}
\label{eq:exp_iz_estimator}
\widehat{\mathcal{I}}_t^{(z)}(k)
\coloneqq
\sum_{m=1}^M \bar{w}_t^{(m)}\left(\frac{1}{S}\sum_{s=1}^S \Big[\log p_m(E_{m,s})-\log p_{\mathrm{mix}}(E_{m,s})\Big]\right),
\qquad
E_{m,s}\overset{\text{iid}}{\sim}\mathcal{N}(\bar e_{t,k}^{\mathrm{pred},(m)},\Sigma_{t,k}^{\mathrm{pred},(m)}).
\end{equation}

\paragraph{Stable evaluation of \(\log p_{\mathrm{mix}}(e)\).}
Compute Gaussian log-densities via
\begin{equation}
\label{eq:exp_logpdf_gauss}
\log \mathcal{N}(e;\mu,\Sigma)
=
-\frac12\left(d_y\log(2\pi)+\log\det(\Sigma)+(e-\mu)^\top \Sigma^{-1}(e-\mu)\right).
\end{equation}
Define \(\ell_m(e)\coloneqq \log \bar{w}_t^{(m)}+\log \mathcal{N}(e;\bar e_{t,k}^{\mathrm{pred},(m)},\Sigma_{t,k}^{\mathrm{pred},(m)})\).
Then compute \(\log p_{\mathrm{mix}}(e)\) by a stable log-sum-exp:
\begin{equation}
\label{eq:exp_logmix_lse}
\log p_{\mathrm{mix}}(e)
=
\log\!\left(\sum_{m=1}^M e^{\ell_m(e)}\right)
=
a(e)+\log\!\left(\sum_{m=1}^M e^{\ell_m(e)-a(e)}\right),
\qquad
a(e)\coloneqq \max_{m\in\{1,\dots,M\}} \ell_m(e).
\end{equation}

\paragraph{Final \((z_t,\mathbf{g}_t)\)-information gain.}
Combine \eqref{eq:exp_ig_zg_chain}, \eqref{eq:exp_ig_g_mode}, and \eqref{eq:exp_iz_estimator}:
\begin{equation}
\label{eq:exp_ig_zg_final}
\widehat{\mathrm{IG}}_t(k)
\coloneqq
\widehat{\mathcal{I}}_t^{(z)}(k)
+
\sum_{m=1}^M \bar{w}_t^{(m)}\,
\frac12\log\det\!\left(\mathbf{I}_{d_y}+\mathbf{H}_{t,k}\Sigma^{(m)}_{g,t\mid t-1}\mathbf{H}_{t,k}^\top(\mathbf{S}^{(m)}_{t,k})^{-1}\right).
\end{equation}
In Algorithm~\ref{alg:router_main}, we use \(\mathrm{IG}_t(k)\) as a shorthand for this computable estimate \(\widehat{\mathrm{IG}}_t(k)\).


% =========================
% Proofs
% =========================
\section{Proofs}
\subsection{Proof of Proposition \ref{prop:cross_update}}
\label{app:proof_cross_update}

\propinfo*

\begin{proof}
			Fix $t$ and $m$, and let $\mathcal{G}_t \coloneqq \sigma(\mathcal{F}_t, I_t, z_t=m)$.
			By assumption, the one-step-ahead predictive pair
			$(e_{t,j}^{\mathrm{pred}}, e_{t,I_t}^{\mathrm{pred}})\mid \mathcal{G}_t$ is jointly Gaussian,
			where each term lies in $\mathbb{R}^{d_y}$.
			Under $\mathcal{G}_t$ the realized observation is $e_t=e_{t,I_t}$, and
			$e_t\mid\mathcal{G}_t \overset{d}{=} e_{t,I_t}^{\mathrm{pred}}\mid\mathcal{G}_t$ (since $e_{t,I_t}^{\mathrm{pred}}$ is exactly the one-step predictive residual that generates $e_{t,I_t}$).
		Let
		\[
		\boldsymbol\mu_j \coloneqq \mathbb{E}[e_{t,j}^{\mathrm{pred}}\mid \mathcal{G}_t], \qquad
		\boldsymbol\mu_I \coloneqq \mathbb{E}[e_{t,I_t}^{\mathrm{pred}}\mid \mathcal{G}_t],
	\]
	and define the predictive covariance and cross-covariance matrices
	\[
	\Sigma_I \coloneqq \mathrm{Cov}(e_{t,I_t}^{\mathrm{pred}}\mid \mathcal{G}_t)\in\mathbb{S}^{d_y}_{++}, \qquad
	\Sigma_{jI} \coloneqq \mathrm{Cov}(e_{t,j}^{\mathrm{pred}}, e_{t,I_t}^{\mathrm{pred}}\mid \mathcal{G}_t)\in\mathbb{R}^{d_y\times d_y}.
	\]
		Assume $\Sigma_I$ is non-singular (e.g., due to additive observation noise with $\mathbf{R}_{m,I_t}\succ \mathbf{0}$).
		For jointly Gaussian vectors, the conditional expectation is given by the standard formula
		\[
		\mathbb{E}[e_{t,j}^{\mathrm{pred}}\mid e_{t,I_t}^{\mathrm{pred}}=e_t, \mathcal{G}_t]
		=
		\boldsymbol\mu_j + \Sigma_{jI}\,\Sigma_I^{-1}\,\bigl(e_t-\boldsymbol\mu_I\bigr).
		\]
			Therefore,
			$\mathbb{E}[e_{t,j}^{\mathrm{pred}}\mid e_t, \mathcal{G}_t]=\boldsymbol\mu_j$ for all values of $e_t$
			if and only if $\Sigma_{jI}=\mathbf{0}$, i.e.,
			$\mathrm{Cov}(e_{t,j}^{\mathrm{pred}}, e_{t,I_t}^{\mathrm{pred}}\mid \mathcal{G}_t)=\mathbf{0}$.
\end{proof}

\subsection{Proof of Proposition \ref{prop:invariance}} \label{app:invariance}

\invariance*

	\begin{proof}
	The statement is a direct consequence of the definition of marginalization.

Write the filtering belief at the end of round $t-1$ (conditioned on the realized history, which we omit from the
	notation) as a joint density over the shared factor and all idiosyncratic states:
	\[
	q_{t-1\mid t-1}\Big(\mathbf{g}_{t-1},(\mathbf{u}_{t-1,\ell})_{\ell\in\mathcal{K}_{t-1}}\Big).
	\]
	Let $\mathcal{K}' \coloneqq \mathcal{K}_{t-1}\setminus P_t$ denote the retained experts and denote
	$\mathbf{u}_{t-1,\mathcal{K}'} \coloneqq (\mathbf{u}_{t-1,\ell})_{\ell\in\mathcal{K}'}$.
	By the definition of a marginal density, the joint marginal of the retained variables under $q_{t-1\mid t-1}$ is
	\begin{equation}
	q_{t-1\mid t-1}\big(\mathbf{g}_{t-1},\mathbf{u}_{t-1,\mathcal{K}'}\big)
	\;=\;
	\int q_{t-1\mid t-1}\big(\mathbf{g}_{t-1},\mathbf{u}_{t-1,\mathcal{K}'},(\mathbf{u}_{t-1,k})_{k\in P_t}\big)\,
	\prod_{k\in P_t} d\mathbf{u}_{t-1,k}.
	\label{eq:marg_def}
	\end{equation}
	On the other hand, the post-pruning belief $q^{\mathrm{pr}(P_t)}_{t-1\mid t-1}$ is \emph{defined} by exactly the same integral:
	\[
	q^{\mathrm{pr}(P_t)}_{t-1\mid t-1}\big(\mathbf{g}_{t-1},\mathbf{u}_{t-1,\mathcal{K}'}\big)
	\;\coloneqq\;
	\int q_{t-1\mid t-1}\big(\mathbf{g}_{t-1},\mathbf{u}_{t-1,\mathcal{K}'},(\mathbf{u}_{t-1,k})_{k\in P_t}\big)\,
	\prod_{k\in P_t} d\mathbf{u}_{t-1,k}.
	\]
	Comparing with \eqref{eq:marg_def} yields
	\[
	q^{\mathrm{pr}(P_t)}_{t-1\mid t-1}\big(\mathbf{g}_{t-1},\mathbf{u}_{t-1,\mathcal{K}'}\big)
	=
	q_{t-1\mid t-1}\big(\mathbf{g}_{t-1},\mathbf{u}_{t-1,\mathcal{K}'}\big),
	\]
	which proves that pruning $P_t$ leaves the joint belief over all retained variables unchanged.

	For the stated consequences, let $\ell\notin P_t$.
	The SLDS time update propagates $(\mathbf{g}_{t-1},\mathbf{u}_{t-1,\ell})$ to $(\mathbf{g}_t,\mathbf{u}_{t,\ell})$
	using the same linear-Gaussian transition under both beliefs. Since the retained marginal
	$q_{t-1\mid t-1}(\mathbf{g}_{t-1},\mathbf{u}_{t-1,\ell})$ is identical before and after pruning, the predictive
	distribution of $(\mathbf{g}_t,\mathbf{u}_{t,\ell})$ is also identical. Because
	$\boldsymbol{\alpha}_{t,\ell}=\mathbf{B}_\ell \mathbf{g}_t+\mathbf{u}_{t,\ell}$ is a measurable function of
	$(\mathbf{g}_t,\mathbf{u}_{t,\ell})$ and $e_{t,\ell}^{\mathrm{pred}}$ follows the emission model given these
	states, the predictive distributions of $\boldsymbol{\alpha}_{t,\ell}$ and $e_{t,\ell}^{\mathrm{pred}}$
	are unchanged by pruning.
	\end{proof}

\subsection{Proof of Proposition \ref{prop:transfer}} \label{app:transfer}

\transfer*

		\begin{proof}
		Fix $t$ and condition on $(\mathcal{F}_t,z_t=m)$.
		Under the factorized one-step predictive belief, for any $j\neq k$ we have the marginal factorization
		\[
		q(\mathbf{g}_t,\mathbf{u}_{t,j},\mathbf{u}_{t,k}\mid \mathcal{F}_t,z_t=m)
		=
		q(\mathbf{g}_t\mid \mathcal{F}_t,z_t=m)\,q(\mathbf{u}_{t,j}\mid \mathcal{F}_t,z_t=m)\,q(\mathbf{u}_{t,k}\mid \mathcal{F}_t,z_t=m),
		\]
		so $\mathbf{g}_t \perp\!\!\!\perp \mathbf{u}_{t,\ell}$ for all $\ell$ and
		$\mathbf{u}_{t,j}\perp\!\!\!\perp \mathbf{u}_{t,k}$ for $j\neq k$.
	Recalling $\boldsymbol{\alpha}_{t,\ell}=\mathbf{B}_\ell \mathbf{g}_t+\mathbf{u}_{t,\ell}$ and using bilinearity of covariance,
	\begin{align*}
	\mathrm{Cov}(\boldsymbol{\alpha}_{t,j},\boldsymbol{\alpha}_{t,k}\mid \mathcal{F}_t,z_t=m)
	&=\mathrm{Cov}(\mathbf{B}_j\mathbf{g}_t+\mathbf{u}_{t,j},\,\mathbf{B}_k\mathbf{g}_t+\mathbf{u}_{t,k}\mid \mathcal{F}_t,z_t=m)\\
	&=\mathrm{Cov}(\mathbf{B}_j\mathbf{g}_t,\,\mathbf{B}_k\mathbf{g}_t\mid \mathcal{F}_t,z_t=m)
	+\mathrm{Cov}(\mathbf{B}_j\mathbf{g}_t,\,\mathbf{u}_{t,k}\mid \mathcal{F}_t,z_t=m)\\
	&\quad+\mathrm{Cov}(\mathbf{u}_{t,j},\,\mathbf{B}_k\mathbf{g}_t\mid \mathcal{F}_t,z_t=m)
	+\mathrm{Cov}(\mathbf{u}_{t,j},\,\mathbf{u}_{t,k}\mid \mathcal{F}_t,z_t=m)\\
	&=\mathbf{B}_j\,\mathrm{Cov}(\mathbf{g}_t,\mathbf{g}_t\mid \mathcal{F}_t,z_t=m)\,\mathbf{B}_k^\top\\
	&=\mathbf B_j\,\Sigma^{(m)}_{g,t|t-1}\,\mathbf B_k^\top,
	\end{align*}
	where $\Sigma^{(m)}_{g,t|t-1}\coloneqq \mathrm{Cov}(\mathbf{g}_t\mid \mathcal{F}_t,z_t=m)$.
	If $\mathbf B_j\Sigma^{(m)}_{g,t|t-1}\mathbf B_k^\top\neq 0$ and the joint predictive law of
	$(\boldsymbol\alpha_{t,j},\boldsymbol\alpha_{t,k})$ is Gaussian, then the pair is not independent, hence
	$\mathcal{I}(\boldsymbol{\alpha}_{t,j};\boldsymbol{\alpha}_{t,k}\mid \mathcal{F}_t,z_t=m)>0$.
	\end{proof}

\section{Experiments Details}
\label{sec:experiments-details}

We provide additional details on the experiments of Section~\ref{section:experiments}, including
experimental setup, hyperparameters, and implementation details.

\paragraph{Compared methods.}
We compare our \textbf{L2D-SLDS} router under bandit feedback to the following baselines.
\emph{(i) Ablation:} L2D-SLDS without the shared global factor (set \(d_g=0\)).
\emph{(ii) Contextual bandits:} LinUCB \citep{li2010contextual} and NeuralUCB
\citep{zhou2020neuralcontextualbanditsucbbased} with more details in \ref{subsection:baselines}.

\paragraph{Metric.}
We report the time-averaged cumulative routing cost over horizon \(T\) (Eq.~\eqref{eq:routing_objective}). Concretely, we
compute the estimate
\(
\hat{J}(\pi) \coloneqq \frac{1}{T}\sum_{t=1}^T C_{t,I_t},
\)
where \(C_{t,I_t}\) is the realized cost of deferring to the selected expert at round \(t\). Lower is better.

\subsection{Baselines} \label{subsection:baselines}

\paragraph{Feedback regimes.}
At round \(t\), the router observes \((\mathbf{x}_t,\mathcal{E}_t)\), chooses \(I_t\in\mathcal{E}_t\),
and then observes \((\vhaty_{t,I_t},\vy_t)\), hence the realized residual \(e_t=e_{t,I_t}\) and realized
cost \(C_t=C_{t,I_t}\), where \(C_{t,k}\coloneqq \psi(e_{t,k})+\beta_k\) and
\(e_{t,k}=\vhaty_{t,k}-\vy_t\) (Appendix~\ref{app:notation}).
\emph{Partial feedback} means only \((\vhaty_{t,I_t},\vy_t)\) is observed after acting.

\paragraph{L2D-SLDS and ablation without \(\mathbf{g}_t\).}
Our method is the model-based router of Algorithm~\ref{alg:router_main} under the generative residual
model of Definition~\ref{def:l2d_slds_emission}:
\(
\boldsymbol{\alpha}_{t,k}=\mathbf{B}_k\mathbf{g}_t+\mathbf{u}_{t,k}
\)
and
\(
e_{t,k}\mid(z_t=m,\mathbf{g}_t,\mathbf{u}_{t,k},\mathbf{x}_t)\sim
\mathcal{N}\!\big(\Phi(\mathbf{x}_t)^\top\boldsymbol{\alpha}_{t,k},\mathbf{R}_{m,k}\big)
\)
(\eqref{eq:alpha_def}--\eqref{eq:residual_emission}).
\textbf{L2D-SLDS w/o \(\mathbf{g}_t\)} is the ablation obtained by setting \(d_g=0\) (equivalently
\(\mathbf{B}_k\mathbf{g}_t\equiv \mathbf{0}\) for all \(k\)), so that
\(\boldsymbol{\alpha}_{t,k}=\mathbf{u}_{t,k}\) and the per-expert predictive residuals are conditionally
independent across experts under the factorized belief (no cross-expert transfer through a shared
factor).

\paragraph{Contextual bandits: LinUCB and NeuralUCB (partial and full feedback).}
Both methods operate on the per-round \emph{cost} \(C_{t,k}\) and are implemented as \emph{lower}
confidence bound (LCB) rules since we minimize cost.
Under \emph{full feedback}, the router observes \(\{C_{t,k}\}_{k\in\mathcal{E}_t}\) regardless of which
expert \(I_t\) was selected. Consequently, the usual exploration--exploitation trade-off disappears:
the choice of \(I_t\) does not affect what data is available for learning, so the exploration bonus
can be set to \(0\) (yielding greedy selection) without sacrificing statistical efficiency. We still
state the LCB form below for a unified presentation.

\subparagraph{LinUCB.}
Fix a feature map \(\varphi:\mathbb{R}^d\to\mathbb{R}^p\) (in our experiments, either raw
\(\mathbf{x}_t\) or an RNN embedding).
Assume a linear model for the conditional mean cost of each expert:
\(
\mathbb{E}[C_{t,k}\mid \mathbf{x}_t]\approx \varphi(\mathbf{x}_t)^\top \boldsymbol{\theta}_k
\).
Maintain ridge statistics per expert \(k\), with ridge parameter \(\lambda>0\).
Under \emph{partial feedback}:
\[
\mathbf{V}_{t,k}\coloneqq \lambda \mathbf{I}_p + \sum_{s<t:\ I_s=k} \varphi(\mathbf{x}_s)\varphi(\mathbf{x}_s)^\top,
\qquad
\mathbf{b}_{t,k}\coloneqq \sum_{s<t:\ I_s=k} \varphi(\mathbf{x}_s)C_s,
\qquad
\widehat{\boldsymbol{\theta}}_{t,k}\coloneqq \mathbf{V}_{t,k}^{-1}\mathbf{b}_{t,k}.
\]
where \(C_s=C_{s,I_s}\) is the realized (queried) cost at round \(s\).  At time \(t\), set
\(
\widehat{C}_{t,k}\coloneqq \varphi(\mathbf{x}_t)^\top\widehat{\boldsymbol{\theta}}_{t,k}
\)
and exploration bonus
\(
u_t(k)\coloneqq \alpha_t\sqrt{\varphi(\mathbf{x}_t)^\top\mathbf{V}_{t,k}^{-1}\varphi(\mathbf{x}_t)}.
\)
The decision rule is
\[
I_t \in \arg\min_{k\in\mathcal{E}_t}\ \widehat{C}_{t,k}-u_t(k).
\]
\emph{Partial feedback LinUCB} updates only the chosen arm \(I_t\) (so only \(C_{t,I_t}=C_t\) is observed).

\subparagraph{NeuralUCB.}
Let \(f_{\boldsymbol{\omega}}(\mathbf{x},k)\) be a neural predictor of the conditional mean cost of expert
\(k\) given \(\mathbf{x}\) (we use a shared encoder with a per-expert head).
Define a parameter-gradient feature (to avoid overloading the shared factor \(\mathbf{g}_t\))
\(
\mathbf{h}_{t,k}\coloneqq \nabla_{\boldsymbol{\omega}} f_{\boldsymbol{\omega}}(\mathbf{x}_t,k)\in\mathbb{R}^{p_\omega}.
\)
Maintain a (regularized) Gram matrix. Under \emph{partial feedback}:
\[
\mathbf{A}_t\coloneqq \lambda \mathbf{I}_{p_\omega}
+
\sum_{s<t} \mathbf{h}_{s,I_s}\mathbf{h}_{s,I_s}^\top.
\]

At time \(t\), set
\(
\widehat{C}_{t,k}\coloneqq f_{\boldsymbol{\omega}}(\mathbf{x}_t,k)
\)
and
\(
u_t(k)\coloneqq \alpha_t\sqrt{\mathbf{h}_{t,k}^\top \mathbf{A}_t^{-1}\mathbf{h}_{t,k}}.
\)
The decision rule is
\[
I_t \in \arg\min_{k\in\mathcal{E}_t}\ \widehat{C}_{t,k}-u_t(k).
\]
The network is trained online by stochastic gradient steps on squared error.
\emph{Partial feedback NeuralUCB} uses the loss \((f_{\boldsymbol{\omega}}(\mathbf{x}_t,I_t)-C_t)^2\) (only \(C_t\) observed).

\paragraph{Oracle baseline.}
The (per-round) oracle chooses the best available expert in hindsight:
\[
I_t^{\mathrm{oracle}} \in \arg\min_{k\in\mathcal{E}_t} C_{t,k}.
\]
This is infeasible under partial feedback because \(C_{t,k}\) is not observed for all \(k\), but we
report it as a lower bound on achievable cumulative cost.


\subsection{Synthetic: Regime-Dependent Correlation and Information Transfer}
\label{sec:exp_synthetic_transfer_appendix}

\paragraph{Design goal.}
We construct a controlled routing instance in which (i) experts are \emph{correlated} in a
regime-dependent way, so that observing one expert should update beliefs about others (information
transfer; Proposition~\ref{prop:cross_update}); and (ii) one expert temporarily disappears and
re-enters, so that the maintained registry \(\mathcal{K}_t\) matters (see Appendix).

\paragraph{Environment (regimes, target, context).}
We use \(M=2\) regimes and deterministic switching in blocks of length \(L=150\) over horizon
\(T=3000\) such as $z_t \coloneqq 1 + \left\lfloor \frac{t-1}{L}\right\rfloor \bmod 2$.
The target follows a regime-dependent AR(1), and the context is the one-step lag:
\begin{equation}
\label{eq:exp_tri_cycle_ts_appendix}
y_t = 0.8\,y_{t-1} + d_{z_t} + \eta_t,\qquad \eta_t\sim\mathcal{N}(0,\sigma_y^2).
\end{equation}
We set the router's context to \(x_t\coloneqq y_{t-1}\).
The regime \(z_t\) is latent to the router: the router observes only \(x_t\) (before acting) and the
single queried prediction \(\hat y_{t,I_t}\) (after acting).

\paragraph{Experts and availability.}
We use \(K=4\) experts indexed \(k\in\{0,1,2,3\}\). Expert \(k=1\) is removed from the available set \(\mathcal{E}_t\)
for a contiguous interval \(t\in[2000,2500]\) and then re-enters. Each expert is a one-step forecaster
\(\hat y_{t,k}=f_k(x_t)\) with a shared slope and expert-specific intercept plus noise:
\begin{equation}
\label{eq:exp_synth_expert_rule_appendix}
\hat y_{t,k} \coloneqq 0.8\,y_{t-1} + b_k + \varepsilon_{t,k}.
\end{equation}
We set \((b_0,b_1,b_2,b_3)=(d_1,d_1,d_2,d_2)\), so experts \(\{0,1\}\) are well-calibrated in regime
\(z_t=1\) and experts \(\{2,3\}\) are well-calibrated in regime \(z_t=2\).

To induce \emph{regime-dependent correlation} under bandit feedback, we generate the expert noises as
\[
\varepsilon_{t,k} \coloneqq s_{t,g(k)} + \tilde\varepsilon_{t,k},
\qquad
g(k)\coloneqq 1+\mathbf{1}\{k\in\{2,3\}\},
\]
with independent components \(s_{t,1},s_{t,2},(\tilde\varepsilon_{t,k})_{k}\) and regime-dependent
variances $s_{t,1}\sim\mathcal{N}(0,\sigma_{z_t,1}^2), s_{t,2}\sim\mathcal{N}(0,\sigma_{z_t,2}^2),
\tilde\varepsilon_{t,k}\sim\mathcal{N}(0,\sigma_{\mathrm{id}}^2)$, where \((\sigma_{1,1}^2,\sigma_{1,2}^2)=(\sigma_{\mathrm{hi}}^2,\sigma_{\mathrm{lo}}^2)\) and
\((\sigma_{2,1}^2,\sigma_{2,2}^2)=(\sigma_{\mathrm{lo}}^2,\sigma_{\mathrm{hi}}^2)\) with
\(\sigma_{\mathrm{hi}}^2\gg\sigma_{\mathrm{lo}}^2\). This makes experts \(\{0,1\}\) strongly
correlated in regime \(1\) and experts \(\{2,3\}\) strongly correlated in regime \(2\). We report the MSE of each expert in
Table~\ref{tab:exp_avg_costs}.


\begin{table}[ht]
\centering
\small
\setlength{\tabcolsep}{4pt}
\caption{Averaged cumulative cost \eqref{eq:routing_objective} on experiment (Section~\ref{sec:exp_synthetic_transfer}).
We report mean \(\pm\) standard error across five runs. Lower is better.}
\label{tab:exp_avg_costs_appendix}
\begin{tabular}{@{}lcc@{}}
\toprule
Method & Averaged Cumulative Cost \\
\midrule
\textbf{L2D-SLDS} & \(\mathbf{13.58 \pm 0.07}\)  \\
L2D-SLDS w/o \(\mathbf{g}_t\) & \(14.68 \pm 0.01\)  \\
\midrule
LinUCB & \(22.94 \pm 0.01\)  \\
NeuralUCB & \(21.92 \pm 0.31\)  \\
Random & \(26.13 \pm 0.25\)   \\
Always expert 0 & \(23.07\)  \\
Always expert 1 & \(28.66\)  \\
Always expert 2 & \(23.05\)  \\
Always expert 3 & \(29.36\)  \\
Oracle & \(9.04\)\\
\bottomrule
\end{tabular}
\end{table}

\textbf{Model Configuration.}
We use $M = 2$ regimes with shared factor dimension $d_g = 1$ and idiosyncratic dimension $d_\alpha = 1$.
The staleness horizon for pruning is $\Delta_{\max} = 500$. We simply run a small warmup of 100 steps before
running L2D-SLDS and UCBs.


\paragraph{Correlation recovery.}
Figure~\ref{fig:exp_synthetic_transfer_regime_estimation} compares the regime-0 loss correlation
structure. The ground truth exhibits a clear block structure: experts \(\{0,1\}\)
form one correlated group while experts \(\{2,3\}\) form another.
Under partial feedback, L2D-SLDS is the only method that reliably recovers this clustering from
partial observations, whereas removing the shared factor \(\mathbf{g}_t\) blurs the separation and
inflates cross-group correlations, consistent with losing cross-expert information transfer. In
contrast, LinUCB/NeuralUCB yield near-degenerate correlation estimates (e.g., overly uniform or
unstable patterns), reflecting that purely discriminative bandit updates do not maintain a coherent
joint belief over experts' latent error processes.

\paragraph{Results and Analysis.}
Table~\ref{tab:exp_avg_costs} shows that \textbf{L2D-SLDS} achieves the lowest routing cost under
partial feedback (\(13.58\pm 0.07\)), improving over LinUCB/NeuralUCB by a wide margin and also
outperforming the best fixed expert. Crucially, it also beats the ablation that removes the shared
factor \(\mathbf{g}_t\) (\(14.68\pm 0.01\)), a \(\approx 7.5\%\) reduction, which directly supports
our central claim: under censoring, modeling a \emph{global} latent component enables
\emph{cross-expert information transfer} from a single queried residual (see Proposition \ref{prop:transfer}). Intuitively, \(\mathbf{g}_t\)
captures regime-dependent common shocks that couple experts; thus, querying one expert updates beliefs
about unqueried experts in a way that contextual bandits (which treat arms largely independently) and
independent per-expert dynamics cannot replicate.

In Appendix, we provide additional experiments that probe this regime-dependent setting in more depth, including detailed
analyses of expert pruning and re-entry.


\begin{figure}[h]
    \centering
    \includegraphics[width=1\textwidth]{figures/expert_structure_all}
    \caption{We report the selection frequency of each expert over time as a function of the underlying regime.
The top figure corresponds to the oracle, while the bottom figure shows our approach evaluated against the baselines.
By construction, experts~0 and~1 perform better in regime~1, whereas experts~2 and~3 perform better in regime~2.
Accordingly, a well-adapted router should select experts~0 and~1 more frequently in regime~1 and experts~2 and~3 more frequently in regime~2.
L2D-SLDS (with and without $g_t$) is the only method that captures this structure, closely matching the oracle’s selection behavior.
In contrast, LinUCB and NeuralUCB fail to adapt their selection frequencies to the regimes.}
    \label{fig:expert_structure_all}
\end{figure}

\subsection{ETTh1}
\label{sec:exp_etth1_appendix}

\paragraph{Environment.}
We evaluate L2D-SLDS on the ETTh1 electricity transformer temperature dataset~\cite{haoyietal-informer-2021}, using the oil
temperature (OT) channel as the target \(y_t\). We run the router over the full horizon (\(T=17420\) hourly observations).
Following the synthetic setup, the router uses a one-step lag as context, \(x_t \coloneqq y_{t-1}\) (with \(x_0=0\)).
There is no observed regime annotation for ETTh1; the router observes only \(x_t\) before acting and, after selecting
\(I_t\in\mathcal{E}_t\), it observes the realized outcome \(y_t\) and the single queried prediction \(\hat y_{t,I_t}\) (hence the
queried residual \(e_{t,I_t}\)).

\paragraph{Experts and availability.}
We consider \(K=6\) fixed experts (Table~\ref{tab:experts_etth}). To stress-test dynamic availability and our pruning/re-birth
mechanism, we enforce time-varying expert sets: the strong multi-lag baseline (Expert~4) is available only on the interval
\(t\in[1000,2000]\), while Expert~0 is unavailable on the same interval. This prevents degenerate ``always-pick-the-best''
policies and forces the router to handle both expert arrival/departure (Expert~4) and temporary unavailability with later
return (Expert~0).

\begin{table}[ht]
\centering
\small
\setlength{\tabcolsep}{4pt}
\caption{Configuration of experts for ETTh1.}
\label{tab:experts_etth_appendix}
\begin{tabular}{@{}lll@{}}
\toprule
Index & Base & Modification \\
\midrule
\textbf{0} & AR(1) & small variance \\
\textbf{1} & AR(1) & large variance \\
\textbf{2} & MLP & trained on early 2/3 of data \\
\textbf{3} & MLP & trained on late 2/3 of data \\
\textbf{4} & AR multi-lag baseline & using lags [1, 24, 168] \\
\textbf{5} & Constant &  always predict 0 \\
\bottomrule
\end{tabular}
\end{table}

\textbf{Model Configuration.}
We use $M = 5$ regimes with shared factor dimension $d_g = 2$ and idiosyncratic dimension $d_\alpha = 1$.
The staleness horizon for pruning is $\Delta_{\max} = 250$. The exploration term considers information gain on both global factor $g$ and regime $z$.
Online EM adaptation is enabled with a sliding window of $W = 600$ and updates every 300 steps.


\begin{table}[ht]
\centering
\small
\setlength{\tabcolsep}{4pt}
\caption{Averaged cumulative cost \eqref{eq:routing_objective} on ETTh1 (Section~\ref{sec:exp_etth1}).
We report the mean $\pm$ standard error over five runs; lower is better.
The averaged cumulative cost is computed both over the full time horizon and, for each expert, only over the periods during which that expert is available.
This explains why Expert~4 attains a low cost despite being available for only a short duration.
Consequently, it is expected that baseline methods exhibit higher averaged costs than Expert~4.}
\label{tab:exp_avg_costs_etth_appendix}
\begin{tabular}{@{}lcc@{}}
\toprule
Method & Averaged Cumulative Cost  \\
\midrule
\textbf{L2D-SLDS} & \(\mathbf{0.80 \pm 0.06}\)  \\
L2D-SLDS w/o \(\mathbf{g}_t\) & \(0.93 \pm 0.08\) \\
\midrule
LinUCB & \(0.84 \pm 0.01\)  \\
NeuralUCB & \(1.09 \pm 0.19\)  \\
Random & \(14.51 \pm 0.73\)  \\
Always expert 0 & \(0.81\)  \\
Always expert 1 & \(1.19\) \\
Always expert 2 & \(0.77\) \\
Always expert 3 & \(1.21\) \\
Always expert 4 & \(0.74\)  \\
Always expert 5 & \(166.65\) \\
Oracle & \(0.24 \pm 0.01\)  \\
\bottomrule
\end{tabular}
\end{table}

\paragraph{Results and analysis.}
Table~\ref{tab:exp_avg_costs_etth_appendix} reports the averaged cumulative routing cost. Under \emph{partial feedback}, L2D-SLDS achieves
the lowest cost among adaptive methods that learn online from bandit feedback (\(0.80\pm 0.06\)), improving over LinUCB
(\(0.84\)) and substantially outperforming NeuralUCB (\(1.09\pm 0.19\)). Most importantly, removing the shared factor
\(\mathbf{g}_t\) degrades performance
(\(0.93\pm 0.08\)), a relative increase of \(\approx 15\%\). This gap is consistent with the role of \(\mathbf{g}_t\) under
censoring: ETTh1 exhibits common shocks (e.g., global load/temperature patterns) that affect multiple experts similarly, so a
shared latent component lets a single queried residual update beliefs about \emph{unqueried} experts via the learned
cross-expert dependence.



\subsection{FRED: Treasury Securities at 10-Year Constant Maturity}
\label{subsection_appendix_fred}

\paragraph{Environment.}
We evaluate on the FRED DGS10 series (10-year U.S.\ Treasury constant-maturity yield)~\cite{FRED_DGS10}, using the daily
observations in \texttt{data/FRED\_DGS10.csv} from 1990-01-02 through 2023-12-29 (\(T=8506\)).
The target is \(y_t\coloneqq \text{DGS10}_t\). The router uses a fixed context vector \(x_t\in\mathbb{R}^{10}\) consisting of
yield lags at \(\{1,5,20,60,120,250\}\) days and calendar features for day-of-week and month encoded as sine/cosine pairs.
We z-score normalize each context dimension using the first 2520 observations.
As in all partial-feedback experiments, at each round \(t\) the router observes \((x_t,\mathcal{E}_t)\), chooses \(I_t\), and
then observes \(\hat y_{t,I_t}\) and \(y_t\) (hence the queried residual \(e_{t,I_t}\)).

\paragraph{Experts.}
We use \(K=4\) ridge-regularized linear autoregressive experts (AR) of the form \(\hat y_{t,k}=w_k^\top x_t+b_k\), each trained
offline on a disjoint historical date range and then deployed across the full evaluation horizon. To avoid a single expert
becoming deterministically dominant, we add mild i.i.d.\ Gaussian prediction noise with standard deviation \(0.03\) to each
expert's output. All experts are available at all times (\(\mathcal{E}_t=\{0,1,2,3\}\)).

\begin{table}[ht]
\centering
\small
\setlength{\tabcolsep}{4pt}
\caption{Configuration of experts for the FRED DGS10 experiment.}
\label{tab:experts_fred_appendix}
\begin{tabular}{@{}lll@{}}
\toprule
Index & Model & Training window \\
\midrule
\textbf{0} & AR (linear ridge on \(x_t\)) & 1990-01-02--2000-12-31 \\
\textbf{1} & AR (linear ridge on \(x_t\)) & 2001-01-01--2007-12-31 \\
\textbf{2} & AR (linear ridge on \(x_t\)) & 2008-01-01--2015-12-31 \\
\textbf{3} & AR (linear ridge on \(x_t\)) & 2016-01-01--2023-12-31 \\
\bottomrule
\end{tabular}
\end{table}

\textbf{Model Configuration.}
We use \(M=4\) regimes with shared factor dimension \(d_g=2\) and idiosyncratic dimension \(d_\alpha=10\) (matching the context
dimension). The staleness horizon is \(\Delta_{\max}=4000\), measurement noise is set to \(R=0.01\), and exploration uses the
information gain over both \(\mathbf{g}_t\) and \(z_t\) (mode \texttt{g\_z}). We disable EM adaptation in this experiment.

\paragraph{Results and analysis.}
Table~\ref{tab:exp_avg_var_l2d_fred} reports the averaged cumulative routing cost. Under partial feedback, \textbf{L2D-SLDS}
achieves the lowest cost among adaptive methods (\(0.004327\pm0.000003\)), improving over LinUCB, NeuralUCB, and random routing.
Removing the shared factor slightly degrades performance (\(0.004411\pm0.000011\)), consistent with shared latent structure
providing additional cross-expert signal when only one residual is observed per round.

\begin{table}[ht]
\centering
\small
\setlength{\tabcolsep}{4pt}
\caption{Averaged cumulative cost \eqref{eq:routing_objective} on the FRED (DGS10) experiment (Appendix~\ref{subsection_appendix_fred}).
We report mean \(\pm\) standard error across five runs; lower is better.}
\label{tab:exp_avg_var_l2d_fred}
\begin{tabular}{@{}lc@{}}
\toprule
Method & Average Cumulative Cost \\
\midrule
\textbf{L2D-SLDS}
& \(\mathbf{0.004327 \pm 0.000003}\) \\
L2D-SLDS w/o \(\mathbf{g}_t\)
& \(0.004411 \pm 0.000011\) \\
\midrule
LinUCB
& \(0.004452 \pm 0.000002\) \\
NeuralUCB
& \(0.004424 \pm 0.000023\) \\
Random
& \(0.004455 \pm 0.000009\) \\
\midrule
Always expert 0
& \(0.004411\) \\
Always expert 1
& \(0.004567\) \\
Always expert 2
& \(0.004505\) \\
Always expert 3
& \(0.004329\) \\
Oracle
& \(0.001754\) \\
\bottomrule
\end{tabular}
\end{table}






\end{document}
