\section{Experiments}
\label{sec:experiments}

We evaluate whether the proposed factorized switching LDS router from
Section~\ref{sec:generative_model} can (i) transfer information across experts under bandit feedback
via the shared factor \(\mathbf{g}_t\) (Proposition~\ref{prop:cross_update}), (ii) track non-stationary
regime structure, and (iii) support predictive scheduling. All results in this section use the
reproducible configuration \texttt{config/exp\_synthetic\_1.yaml}; diagnostics are saved in
\texttt{out/tri\_cycle\_corr/}.

\subsection{Synthetic Tri-Cycle with Regime-Dependent Correlation}
\label{sec:exp_tri_cycle_corr}

\begin{figure*}[t]
  \centering
  \IfFileExists{out/tri_cycle_corr/regime_posterior.pdf}{%
    \includegraphics[width=0.32\textwidth]{out/tri_cycle_corr/regime_posterior.pdf}%
  }{%
    \includegraphics[width=0.32\textwidth]{../out/tri_cycle_corr/regime_posterior.pdf}%
  }\hfill
  \IfFileExists{out/tri_cycle_corr/corr_pairs.pdf}{%
    \includegraphics[width=0.32\textwidth]{out/tri_cycle_corr/corr_pairs.pdf}%
  }{%
    \includegraphics[width=0.32\textwidth]{../out/tri_cycle_corr/corr_pairs.pdf}%
  }\hfill
  \IfFileExists{out/tri_cycle_corr/transfer_probe.pdf}{%
    \includegraphics[width=0.32\textwidth]{out/tri_cycle_corr/transfer_probe.pdf}%
  }{%
    \includegraphics[width=0.32\textwidth]{../out/tri_cycle_corr/transfer_probe.pdf}%
  }
  \caption{\textbf{Tri-cycle correlation diagnostics (config \texttt{exp\_synthetic\_1.yaml}).}
  \emph{Left:} regime posterior vs.\ true alternating regime schedule. \emph{Middle:} tracking of
  regime-dependent correlations for the key expert pairs. \emph{Right:} transfer probe for expert
  \(k=1\) (shaded region indicates unavailability), illustrating correlation-mediated belief updates
  through the shared factor \(\mathbf{g}_t\) under censored feedback.}
  \label{fig:tri_cycle_corr}
\end{figure*}

\paragraph{Environment.}
We study a controlled synthetic routing instance in the scalar case (\(d_y=1\)) where the
\emph{routing problem} is as in Section~\ref{sec:preliminaries} (bandit feedback, only the queried
expert is observed), while the \emph{experts themselves} are generated by an explicit latent-factor
time-series model (full specification in \texttt{Paper/Experiments\_Synthetic.tex} and
\texttt{config/exp\_synthetic\_1.yaml}).

\paragraph{Time series.}
Regimes alternate deterministically in blocks of length \(L\) over horizon \(T\),
\(
z_t=\lfloor t/L\rfloor \bmod 2
\),
and the target follows a regime-dependent AR(1) with context \(x_t=y_{t-1}\):
\begin{equation}
\label{eq:exp_tri_cycle_ts}
y_t = 0.8\,y_{t-1} + d_{z_t} + \eta_t,\qquad \eta_t\sim\mathcal{N}(0,\sigma_y^2),
\qquad x_t = y_{t-1}.
\end{equation}

\paragraph{Expert generator (bias + correlation).}
Each expert \(k\in\{0,1,2,3\}\) is parameterized by a regime-dependent bias mean \(b_{m,k}\) and a
shared-factor loading \(B_k\in\mathbb{R}^{1\times 2}\). A shared latent factor
\(\mathbf{g}_t\in\mathbb{R}^2\) and idiosyncratic states \(u_{t,k}\in\mathbb{R}\) evolve as
\[
\mathbf{g}_t=A_g^{(z_t)}\mathbf{g}_{t-1}+\xi_t,\ \xi_t\sim\mathcal{N}(0,Q_g^{(z_t)}),\qquad
u_{t,k}=b_{z_t,k}+A_u^{(z_t)}(u_{t-1,k}-b_{z_{t-1},k})+\nu_{t,k},\quad
\nu_{t,k}\sim\mathcal{N}(0,Q_u^{(z_t)}).
\]
Using the main-text residual convention \(e_{t,k}=\hat y_{t,k}-y_t\) and \(\phi(x_t)=x_t=y_{t-1}\),
we generate each expert's prediction as an AR(1)-style base term plus an expert-specific
error:
\begin{equation}
\label{eq:exp_tri_cycle_experts}
\hat y_{t,k}
= y_t + e_{t,k}
= \big(0.8\,y_{t-1}+d_{z_t}+\eta_t\big)\;-\;y_{t-1}\big(B_k\mathbf{g}_t+u_{t,k}\big)\;+\;\tilde\varepsilon_{t,k},
\end{equation}
where \(\tilde\varepsilon_{t,k}\) is mean-zero expert-specific Gaussian noise (see
\texttt{Paper/Experiments\_Synthetic.tex}). Equivalently, this is an AR(1)-style form
\(\hat y_{t,k}=a_{t,k}\,y_{t-1}+d_{z_t}+\eta_t+\tilde\varepsilon_{t,k}\) with a time-varying effective
slope \(a_{t,k}\coloneqq 0.8-(B_k\mathbf{g}_t+u_{t,k})\).
Equivalently, the realized routing residual satisfies
\(
e_{t,k}=\hat y_{t,k}-y_t=-y_{t-1}(B_k\mathbf{g}_t+u_{t,k})+\tilde\varepsilon_{t,k}.
\)
This construction makes expert \emph{accuracy} depend on the bias magnitudes \(b_{m,k}\) and induces
within-regime \emph{correlation} through the shared factor:
\(\mathrm{Cov}(e_{t,i},e_{t,j}\mid z_t=m,x_t)=x_t^2\,B_i Q_g^{(m)} B_j^\top\).
In our configuration, \(B_0=B_1=[1,0]\), \(B_2=B_3=[0,1]\), and \(Q_g^{(m)}\) swaps its dominant
component across regimes, so \((0,1)\) is the most correlated pair in regime \(0\) and \((2,3)\) in
regime \(1\); bias means are set to \(b_{0,\cdot}=(0.2,0.3,1.6,1.7)\) and
\(b_{1,\cdot}=(1.6,1.7,0.8,0.9)\), making \(\{0,1\}\) optimal in regime \(0\) and \(\{2,3\}\) optimal
in regime \(1\).

\paragraph{Routing protocol and availability.}
The environment reveals \((x_t,\mathcal{E}_t)\), the router selects \(I_t\in\mathcal{E}_t\), and only
\(\hat y_{t,I_t}\) (hence \(e_{t,I_t}\)) is observed. To stress dynamic feasibility, expert \(k=1\) is
unavailable for \(t\in[2000,2500]\). We evaluate squared loss with zero query fees
(\(\beta_k\equiv 0\)).

\paragraph{Methods and baselines.}
Our method is the factorized SLDS router with context-dependent transition logits
\eqref{eq:context_transitions} and IDS exploration using \((z_t,\mathbf{g}_t)\)-information
(Section~\ref{sec:exploration}). We compare to: (i) an ablation that removes the shared factor
(\(\mathbf{g}_t\equiv \mathbf{0}\), eliminating correlation-mediated transfer), and (ii) bandit
baselines (LinUCB, NeuralUCB) and learning-to-defer style policies (L2D and sliding-window L2D).
For interpretability diagnostics, we additionally include a ``full-feedback'' variant that observes
all expert losses (not available to the router in the main protocol).

\paragraph{Diagnostics.}
We evaluate (a) regime tracking accuracy and switch delay; (b) correlation recovery for the key
pairs \((0,1)\) and \((2,3)\); (c) recovery of the shared factor up to permutation/sign; and (d) a
transfer probe measuring whether unobserved experts are updated through \(\mathbf{g}_t\) when they
are not queried or temporarily unavailable. For predictive scheduling, we run horizon planning with
\(H=50\) starting at \(t_0=1650\) using \(100\) Monte Carlo context scenarios and a relative
coverage threshold \(\delta=0.05\) (config \texttt{horizon\_planning}).

\paragraph{Results.}
Figure~\ref{fig:tri_cycle_corr} summarizes the main effects. Despite highly censored feedback, the
router tracks regime changes with small delay (median \(0\) steps; mean \(0.58\)) and achieves regime
accuracy \(0.72\) over \(t=1,\dots,2999\). The estimated regime-dependent correlations follow the
true pairings: correlation MAE is \(0.075\) in regime \(0\) and \(0.249\) in regime \(1\), with
predicted correlations remaining high for the active pair. The shared factor is identifiable from
single-step residual observations: the recovered \(\mathbf{g}_t\) components match the ground truth
with mean absolute correlation \(0.88\). Finally, the transfer probe shows that when expert \(1\) is
unavailable (shaded region), the \(\mathbf{g}_t\)-based model continues to revise its belief about
expert \(1\)'s loss using observations from correlated experts, while the no-\(\mathbf{g}_t\) ablation
cannot propagate such updates. In horizon planning, the induced demand concentrates on a single
expert across the horizon (active-set size \(\approx 1\) with near-unit captured mass), indicating
that the posterior predictive distribution is sufficiently sharp for stable scheduling in this
setting.
