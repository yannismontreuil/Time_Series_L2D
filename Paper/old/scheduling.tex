% =========================
% Predictive Scheduling Details
% =========================

\section{Predictive Scheduling with Relative-Thresholded Sets}
\label{app:planning_relative_threshold}

We collect the scheduling notation and the relative-threshold rule used by the planner.
At decision time $t$, the planner has access to the maintained registry $\mathcal{K}_t$ and
feasibility constraints $\mathcal{E}^{\mathrm{feas}}_{t+h\mid t}=\mathcal{K}_t\cap\mathcal{A}_{t+h}$.
Given a forecast trajectory $\mathbf{x}_{t+1:t+H}$, the open-loop planner selects
\begin{equation}
\widehat{I}_{t+h}(\mathbf{x}_{t+1:t+h})
\in
\arg\min_{k\in\mathcal{E}^{\mathrm{feas}}_{t+h\mid t}} C_{t+h,k}^{\dagger}(\mathbf{x}_{t+1:t+h}),
\qquad h=1,\dots,H,
\end{equation}
where $C_{t+h,k}^{\dagger}$ is the predicted cost obtained by propagating the time-$t$ belief without
any measurement updates. If $\mathcal{E}^{\mathrm{feas}}_{t+h\mid t}=\varnothing$, we set
$\widehat{I}_{t+h}=\varnothing$. This defines the random planned winner $\widehat{I}_{t+h}$ under the
scenario distribution $p(\mathbf{X}_{t+1:t+H}\mid\mathcal{F}_t)$ and yields the time-marginal demand
\begin{equation}
\rho_{t,h}(k)\coloneqq \mathbb{P}\!\left(\widehat{I}_{t+h}=k\mid\mathcal{F}_t\right),\qquad
k\in\mathcal{K}_t.
\end{equation}

We summarize the demand with a relative-thresholded active set. Let
\(
\rho^{\max}_{t,h}\coloneqq \max_{k\in\mathcal{E}^{\mathrm{feas}}_{t+h\mid t}}\rho_{t,h}(k)
\)
(with $\rho^{\max}_{t,h}=0$ when the feasible set is empty). For $\delta\in[0,1)$, define
\begin{equation}
\mathcal{S}_{t,h}(\delta)
\coloneqq
\left\{k\in\mathcal{E}^{\mathrm{feas}}_{t+h\mid t}:\rho_{t,h}(k)\ge (1-\delta)\rho^{\max}_{t,h}\right\}.
\end{equation}
This rule keeps the on-call experts whose probability is within a multiplicative gap of the most
likely expert. It is scale-free, does not depend on the number of experts, and is never empty when
$\mathcal{E}^{\mathrm{feas}}_{t+h\mid t}\neq\varnothing$. The extremes are
$\delta=0$ (only maximizers) and $\delta\uparrow 1$ (all feasible experts). The mapping is monotone:
if $0\le \delta_1\le \delta_2<1$ then $\mathcal{S}_{t,h}(\delta_1)\subseteq\mathcal{S}_{t,h}(\delta_2)$.

\paragraph{Monte Carlo plug-in estimator.}
With $N$ sampled scenarios, we estimate $\rho_{t,h}(k)$ by
\(
\hat{\rho}_{t,h}(k)=\frac{1}{N}\sum_{n=1}^N\mathbf{1}\{\widehat{I}^{(n)}_{t+h}=k\}
\)
and define the plug-in max
\(
\hat{\rho}^{\max}_{t,h}=\max_{k\in\mathcal{E}^{\mathrm{feas}}_{t+h\mid t}}\hat{\rho}_{t,h}(k)
\).
The estimated active set is
\begin{equation}
\widehat{\mathcal{S}}_{t,h}(\delta)
\coloneqq
\left\{k\in\mathcal{E}^{\mathrm{feas}}_{t+h\mid t}:\hat{\rho}_{t,h}(k)\ge (1-\delta)\hat{\rho}^{\max}_{t,h}\right\}.
\end{equation}
The resulting trajectory $\widehat{\mathcal{S}}_{t:t+H}(\delta)$ is the on-call plan reported by the
planner.
