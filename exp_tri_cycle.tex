%! Author = yannis
%! Date = 09/01/2026

% Preamble
\documentclass[11pt]{article}

% Packages
\usepackage{amsmath}

% Document
\begin{document}

  \subsection{Tri-Cycle Correlation Experiment}
  \label{sec:tri_cycle_corr}

  \paragraph{Goal.}
  We design a synthetic experiment to isolate the two properties emphasized by the
  factorized SLDS: (i) correlation-mediated information transfer across experts,
  and (ii) predictive scheduling under regime-dependent availability.

  \paragraph{Regime dynamics.}
  We use $M=3$ regimes and a deterministic cycle
  $z_t \in \{1,2,3\}$ with pattern
  $1 \to 2 \to 3 \to 2 \to 3 \to 1$ repeated in blocks of fixed length.
  The observed target follows an AR(1) with regime-dependent drift,
  \begin{equation}
  y_t = 0.8\,y_{t-1} + \mu_{z_t} + \varepsilon_t,
  \quad
  \varepsilon_t \sim \mathcal{N}(0,\sigma^2).
  \end{equation}
  The context is the lagged target, $x_t = y_{t-1}$.

  \paragraph{Experts and correlation structure.}
  We define $K=5$ experts with linear predictors
  $\widehat{y}_{t,k} = a_{z_t,k}\,x_t + b_{z_t,k} + \eta_{t,k}$.
  To encode regime-specific correlations, we inject a shared perturbation into
  specific expert pairs within each regime:
  \begin{align}
  \text{Regime 1:} &\quad (k=1,2) \text{ share a common noise component},\\
  \text{Regime 2:} &\quad (k=3,4) \text{ share a common noise component},\\
  \text{Regime 3:} &\quad (k=5,1) \text{ share a common noise component}.
  \end{align}
  The remaining experts receive independent noise. This makes the residuals
  $\{e_{t,k}\}$ correlated within the specified pair for the active regime,
  while correlations change as regimes shift.

  \paragraph{Regime-dependent availability.}
  We further impose known availability constraints:
  expert $5$ is feasible only in regimes $2$ and $3$,
  and expert $4$ is feasible only in regimes $1$ and $2$.
  Formally, for each lookahead $h$, the feasible set
  $\mathcal{E}^{\mathrm{feas}}_{t+h\mid t}$ is computed by intersecting the
  working registry with these regime-dependent availability rules.

  \paragraph{Why this setting is diagnostic.}
  This construction stresses the core claims of the factorized model:
  (1) correlations change with regime, so information transfer should be
  contextual rather than static; (2) partial feedback is informative because
  correlated experts should be updated even when not queried; and (3) predictive
  scheduling must adapt to both regime uncertainty and feasibility constraints.

  \paragraph{What to observe.}
  We recommend reporting:
  \begin{enumerate}
      \item \textbf{Correlation recovery:} estimated cross-expert correlations
      (or shared-factor loadings) should track the true pairings across regimes.
      \item \textbf{Partial-feedback advantage:} in partial feedback, the
      factorized model should still update unqueried correlated experts, yielding
      lower loss than baselines that treat experts as independent.
      \item \textbf{Regime transitions:} posterior regime weights should react
      quickly to the cycle and the correlated pair should switch accordingly.
      \item \textbf{Scheduling structure:} the active sets
      $\widehat{\mathcal{S}}_{t,h}(\delta)$ should (i) respect feasibility and
      (ii) expand near regime transitions when scenario uncertainty spreads mass
      across multiple correlated candidates.
  \end{enumerate}

  \paragraph{Implementation details.}
  The experiment is reproducible via the config file
  \texttt{config/config\_tri\_cycle\_corr.yaml}. We set
  \texttt{environment.setting = tri\_cycle\_corr} with
  \texttt{num\_experts = 5}, \texttt{num\_regimes = 3}, and total horizon $T$.
  The regime cycle is controlled by
  \texttt{environment.tri\_cycle.regime\_pattern} and
  \texttt{environment.tri\_cycle.regime\_block\_len}.
  Regime drifts $\mu_{z_t}$ are set by
  \texttt{environment.tri\_cycle.drift\_levels}, while observation noise uses
  \texttt{environment.noise\_scale}.
  Expert correlations are induced by shared noise with scales
  \texttt{shared\_noise\_scale} and \texttt{indiv\_noise\_scale}.
  Regime-dependent expert accuracy is encoded through
  \texttt{biases\_by\_regime} (and optionally \texttt{slopes\_by\_regime}).
  Note that the implementation uses 0-based expert indices, so paper experts
  $1$--$5$ correspond to code indices $0$--$4$.
  Availability constraints are enforced at the environment level:
  expert 4 (index 3) is available only in regimes 1 and 2, and expert 5 (index 4)
  is available only in regimes 2 and 3.

  If you want, I can also add a short paragraph on the Monte Carlo scheduling setup for this experiment (i.e., the specific \texttt{horizon_planning} knobs and how they reflect
  Section~\ref{sec:mc_planning}).

\end{document}